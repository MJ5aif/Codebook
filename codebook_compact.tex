\documentclass[10pt,twocolumn]{article}

\usepackage[margin=1.0cm,includehead]{geometry}
\usepackage{titlesec}
\usepackage{xcolor}
\usepackage{hyperref}
\usepackage{listings}
\usepackage{multicol}
\usepackage{fancyhdr}
\usepackage[T1]{fontenc} % Ensures proper character encoding
\usepackage{textcomp} % For text symbols
\usepackage[space=true]{accsupp} % For copyable spaces

% Header: page number (right) and team name (left) on every page
\setlength{\headheight}{14pt}
\pagestyle{fancy}
\fancyhf{}
\fancyhead[R]{\thepage}
\fancyhead[L]{\textbf{5U5T Doubt Forces}}
\fancyfoot{}
\renewcommand{\headrulewidth}{0pt}
\fancypagestyle{plain}{%
  \fancyhf{}
  \fancyhead[R]{\thepage}
  \fancyhead[L]{\textbf{5U5T Doubt Forces}}
  \renewcommand{\headrulewidth}{0pt}
}

% Command for copyable space
\newcommand{\copyablespace}{\BeginAccSupp{method=hex,unicode,ActualText=00A0}\hphantom{x}\EndAccSupp{}}

\definecolor{codebg}{rgb}{0.95,0.95,0.95}
\definecolor{keyword}{rgb}{0.0,0.0,1.0}    % blue
\definecolor{type}{rgb}{0.0,0.5,0.0}       % green
\definecolor{stl}{rgb}{0.58,0.0,0.82}      % purple
\definecolor{comment}{rgb}{0.0,0.5,0.0}    % green
\definecolor{string}{rgb}{0.8,0.42,0.24}   % orange
\definecolor{number}{rgb}{0.63,0.13,0.94}  % purple
\definecolor{function}{rgb}{0.0,0.5,1.0}   % cyan for functions
\definecolor{operator}{rgb}{1.0,0.0,1.0}   % magenta for operators

\lstdefinelanguage{cpp}{
  morekeywords=[1]{cin,cout,ios,alignas,std,alignof,asm,auto,break,case,catch,class,const,constexpr,define,memset,
  continue,decltype,default,delete,do,else,enum,explicit,export,extern,false,for,
  friend,if,inline,mutable,namespace,new,noexcept,nullptr,operator,private,
  protected,public,register,return,static,struct,switch,template,this,throw,true,
  try,typedef,typeid,typename,using,virtual,void,volatile,while},
  morekeywords=[2]{stdin,stdout,bool,char,double,float,int,int32_t, int64_t,long,short,signed,unsigned,void,uint64_t},
  morekeywords=[3]{array,ordered_set,vector,map,set,unordered_map,unordered_set,string,pair,tuple},
  morekeywords=[4]{foo,cross,freopen,dot,dist,add,maxxor,insert,bxor,update,query,combine,init,find,merge,NULL,custom_hash,rng,z_func,manacher_odd,suffix_array,BitTrie,Hash,Hashing,BIT,BIT2D,RangeBIT,upd,sum},
  sensitive=true,
  morecomment=[l]{//},
  morecomment=[s]{/*}{*/},
  morestring=[b]",
  morestring=[b]',
  numbers=left,
  numberstyle=\tiny\color{number},
}

\lstset{
  language=cpp,
  basicstyle=\ttfamily\footnotesize\color{black},
  keywordstyle=[1]\color{keyword}\bfseries,
  keywordstyle=[2]\color{type}\bfseries,
  keywordstyle=[3]\color{stl}\bfseries,
  keywordstyle=[4]\color{function}\bfseries,
  commentstyle=\color{comment}\itshape,
  stringstyle=\color{string},
  backgroundcolor=\color{codebg},
  numbers=left,
  numberstyle=\tiny\color{gray},
  stepnumber=1,
  numbersep=8pt,
  xleftmargin=2em,
  framexleftmargin=2em,
  framesep=2pt,
  frame=single,
  rulecolor=\color{gray},
  keepspaces=true,
  columns=fullflexible,
  breaklines=true,
  aboveskip=1pt,
  belowskip=1pt,
  showstringspaces=false,
  commentstyle=\color{comment}\itshape, % Green, italic comments
  literate={*}{{\color{operator}\char42}}1
           {-}{{\color{operator}\char45}}1
           {+}{{\color{operator}+}}1
           {=}{{\color{operator} = }}1
           {==}{{\color{operator}== }}1
           {!}{{\color{operator}!}}1
           {.}{{\color{operator}.}}1
           % \literate={^}{{\color{operator}\^{}}}1
           {<}{{\color{operator}<}}1
           {:}{{\color{operator}:}}1
           {>}{{\color{operator}>}}1
           {&}{{\color{operator}\&}}1
           {|}{{\color{operator}|}}1
           {\ }{{\copyablespace}}1
}

\titlespacing*{\section}{0pt}{1pt}{1pt}
\titlespacing*{\subsection}{0pt}{1pt}{1pt}

\title{TEAM NOTEBOOK --- Doubt Forces (SUST)}
\author{MJ5aif}
\date{\today}

\makeatletter
\renewcommand{\@maketitle}{%
  \newpage
  \null
  \vskip -10pt % Adjust this value to reduce space (negative to decrease)
  \begin{center}
    \let \footnote \thanks
    {\LARGE \@title \par}
    \vskip 1.5em % Space between title and author
    {\large
      \lineskip .5em%
      \begin{tabular}[t]{c}
        \@author
      \end{tabular}\par}
    \vskip 1em % Space between author and date
    {\large \@date}
  \end{center}
  \par
  \vskip 1em % Space after date
}
\makeatother

\begin{document}
% Title + 3-column ToC on the FIRST page (no blank first page)
\begingroup
\onecolumn

\makeatletter
% Temporarily disable the forced page break inside your \@maketitle
\let\saved@newpage\newpage
\let\newpage\relax
\makeatother

% Optional tightening (leave commented if you don't need it)
\vspace*{-0.5em}
\maketitle
\vspace*{-0.75em}

\makeatletter
% Restore normal behavior
\let\newpage\saved@newpage
\makeatother

% Three-column ToC
\setlength{\columnsep}{10pt}
\setlength{\columnseprule}{0pt}
\begin{multicols}{3}
\footnotesize
\tableofcontents
\end{multicols}

\endgroup
% Horizontal line to separate context and content
\vspace*{-0.5em} 
\noindent\rule{\textwidth}{0.4pt} 
\vspace*{0.5em} 

% Main content in 2 columns immediately after line
\setlength{\columnsep}{10pt} % adjust spacing between columns
\begin{multicols}{2}

\section{Templates}
\subsection{BUILD}
\begin{lstlisting}
{
"cmd" : ["g++ -std=c++20 $file_name -o $file_base_name && timeout 5s ./$file_base_name<input.txt>output.txt"],
"selector" : "source.cpp",
"shell": true,
}
\end{lstlisting}

\subsection{STD and Fast IO}
\begin{lstlisting}
freopen("mj5.in", "r", stdin);  freopen("mj5.out", "w", stdout);
ios::sync_with_stdio(false); cin.tie(NULL); cout.tie(NULL);
\end{lstlisting}

\subsection{VS Code Snippet}
\begin{lstlisting}
{
  "snippetName": {
    "scope": "",
    "prefix": "TabTrigger", // replace it 
    "body": [
      "", // every lines must inside->  "",
    ],
    "description": ""
  }
}
// to genrate lines with "", bash command:
cat mj5.cpp | sed 's/\\/\\\\/g; s/"/\\"/g; s/^/"/; s/$/",/'      
\end{lstlisting}
\subsection{Pragma && Debug.h}
\begin{lstlisting}
#pragma comment(linker, "/stack:200000000")
#pragma GCC optimize("Ofast")
#pragma GCC optimize("unroll-loops")
#pragma GCC optimize("-O3")
#pragma GCC optimize("fast-math")
#pragma GCC optimize("no-stack-protector")
#pragma GCC target("avx2,bmi,bmi2,lzcnt,sse,sse2,sse3,ssse3,
sse4,popcnt,abm,mmx,avx,tune=native")
#define see(x) cerr << #x << ": " << x << nl
template<typename... Args>
void debug(Args... args){
    ((cerr << " " << args), ...) << "\n";
}
\end{lstlisting}

\subsection{Ordered Set}
\begin{lstlisting}
// find_by_order() -> iterator // order_of_key() -> int
// inside main function Just declare ordered_set
// change Type by replacing int
#include <ext/pb_ds/assoc_container.hpp> 
#include <ext/pb_ds/tree_policy.hpp> 
using namespace __gnu_pbds;   
#define ordered_set tree<int, null_type,less<int>, rb_tree_tag,tree_order_statistics_node_update>
\end{lstlisting}

\subsection{Custom Hash}
\begin{lstlisting}
struct custom_hash{
static uint64_t sm64(uint64_t x){x+=0x9e3779b97f4a7c15; x=(x^(x>>30))*0xbf58476d1ce4e5b9; x=(x^(x>>27))*0x94d049bb133111eb; return x^(x>>31);} // must
size_t operator()(uint64_t x) const{static const uint64_t R=chrono::steady_clock::now().time_since_epoch().count(); return sm64(x+R);} // int
template<class T1,class T2> size_t operator()(const pair<T1,T2>&p) const{size_t h1=(*this)((uint64_t)p.first),h2=(*this)((uint64_t)p.second); return h1^(h2+0x9e3779b9+(h1<<6)+(h1>>2));} // pair<int,int>
template<class...Ts> size_t operator()(const tuple<Ts...>&t) const{size_t s=0; apply([&](const auto&...e){((s^=(*this)((uint64_t)e)+0x9e3779b9+(s<<6)+(s>>2)),...);},t); return s;} // tuple<int>
template<class T> size_t operator()(const vector<T>&v) const{size_t s=v.size(); for(auto&x:v) s^=(*this)((uint64_t)x)+0x9e3779b9+(s<<6)+(s>>2); return s;}
}; // vector<int> 
\end{lstlisting}

\subsection{Random Number Genrator}
\begin{lstlisting}
int64_t rng(){
   static std::mt19937 gen(std::chrono::steady_clock::now().time_since_epoch().count());
   return std::uniform_int_distribution<int>(1, INT64_MAX)(gen);
}
\end{lstlisting}

\section{Strings}
\subsection{Z-Algo}
\begin{lstlisting}
vector<int> z_func(string &s){
   int n=s.size(),l=0,r=0;    vector<int> z(n);
   for(int i=1;i<n;i++){
      if(i<r)     z[i]=min(r-i,z[i-l]);
      while(i+z[i]<n && s[z[i]]==s[i+z[i]])   z[i]++;
      if(i+z[i]>r){  l=i;     r=i+z[i];   }
   }
   return z;
}
\end{lstlisting}

\subsection{Manacher's Algo}
\begin{lstlisting}
vector<int> manacher_odd(string s) {
    int n = s.size();
    s = "$" + s + "^";
    vector<int> p(n + 2);
    int l = 0, r = 1;
    for(int i = 1; i <= n; i++) {
        p[i] = min(r - i, p[l + (r - i)]);
        while(s[i - p[i]] == s[i + p[i]]) {
            p[i]++;
        }
        if(i + p[i] > r) {
            l = i - p[i], r = i + p[i];
        }
    }
    return vector<int>(begin(p) + 1, end(p) - 1);
}    
\end{lstlisting}

\subsection{Basic Suffix Array}
\begin{lstlisting}
void suffix_array(string s){
   s+="$";  int32_t n=s.size(),N=max(n,int32_t(260));
   vector<int32_t> sa(n),R(n);
   for(int32_t i=0;i<n;i++)  sa[i]=i,  R[i]=s[i];
   for(int32_t k=0;k<n;k?(k<<=1):k++){
      vector<int32_t> nR(n),cnt(N);    auto nsa=sa;
      for(int32_t i=0;i<n;i++) nsa[i]=(nsa[i]-k+n)%n,  cnt[R[i]]++;
      for(int32_t i=1;i<N;i++) cnt[i]+=cnt[i-1];
      for(int32_t i=n-1;i>=0;i--) sa[--cnt[R[nsa[i]]]]=nsa[i];
      for(int32_t i=1,r=0;i<n;i++){
         r+=(R[sa[i]]!=R[sa[i-1]] || R[(sa[i]+k)%n]!=R[(sa[i-1]+k)%n]);
         nR[sa[i]]=r;
      }  
      swap(R,nR);
   }
   vector<int32_t> lcp(n-1);
   for(int32_t i=0,k=0;i<n-1;i++){
      int32_t pi=R[i];
      int32_t j=sa[pi-1];
      while(i+k<n && j+k<n && s[i+k]==s[j+k])   k++;
      lcp[pi-1]=k;
      if(k) k--;
   }
}
\end{lstlisting}

\subsection{Suffix Array}
\begin{lstlisting}
const int N=4e5+9,LG=18;
void induced_sort(const vector<int>& vec,int val_range,vector<int>& SA,const vector<bool>& sl,const vector<int>& lms_idx){
  vector<int>l(val_range,0),r(val_range,0);
  for(int c:vec){if(c+1<val_range)++l[c+1];++r[c];}
  partial_sum(l.begin(),l.end(),l.begin());
  partial_sum(r.begin(),r.end(),r.begin());
  fill(SA.begin(),SA.end(),-1);
  for(int i=lms_idx.size()-1;i>=0;--i) SA[--r[vec[lms_idx[i]]]]=lms_idx[i];
  for(int i:SA) if(i>=1&&sl[i-1]) SA[l[vec[i-1]]++]=i-1;
  fill(r.begin(),r.end(),0);
  for(int c:vec) ++r[c];
  partial_sum(r.begin(),r.end(),r.begin());
  for(int k=SA.size()-1,i=SA[k];k>=1;--k,i=SA[k]) if(i>=1&&!sl[i-1]) SA[--r[vec[i-1]]]=i-1;
}
vector<int> SA_IS(const vector<int>& vec,int val_range){
  int n=vec.size();
  vector<int>SA(n),lms_idx;
  vector<bool>sl(n);
  sl[n-1]=false;
  for(int i=n-2;i>=0;--i){sl[i]=(vec[i]>vec[i+1]||(vec[i]==vec[i+1]&&sl[i+1]));if(sl[i]&&!sl[i+1]) lms_idx.push_back(i+1);}
  reverse(lms_idx.begin(),lms_idx.end());
  induced_sort(vec,val_range,SA,sl,lms_idx);
  vector<int>new_lms_idx(lms_idx.size()),lms_vec(lms_idx.size());
  for(int i=0,k=0;i<n;++i) if(!sl[SA[i]]&&SA[i]>=1&&sl[SA[i]-1]) new_lms_idx[k++]=SA[i];
  int cur=0;SA[n-1]=cur;
  for(size_t k=1;k<new_lms_idx.size();++k){
    int i=new_lms_idx[k-1],j=new_lms_idx[k];
    if(vec[i]!=vec[j]){SA[j]=++cur;continue;}
    bool flag=false;
    for(int a=i+1,b=j+1;;++a,++b){
      if(vec[a]!=vec[b]){flag=true;break;}
      if((!sl[a]&&sl[a-1])||(!sl[b]&&sl[b-1])){flag=!((!sl[a]&&sl[a-1])&&(!sl[b]&&sl[b-1]));break;}
    } SA[j]=(flag?++cur:cur);
  }
  for(size_t i=0;i<lms_idx.size();++i) lms_vec[i]=SA[lms_idx[i]];
  if(cur+1<(int)lms_idx.size()){
    auto lms_SA=SA_IS(lms_vec,cur+1);
    for(size_t i=0;i<lms_idx.size();++i) new_lms_idx[i]=lms_idx[lms_SA[i]];
  } induced_sort(vec,val_range,SA,sl,new_lms_idx);
  return SA;
}
vector<int> suffix_array(const string& s,const int LIM=128){
  vector<int>vec(s.size()+1);
  copy(begin(s),end(s),begin(vec));
  vec.back()='!'; auto ret=SA_IS(vec,LIM);
  ret.erase(ret.begin());
  return ret;
}
struct SuffixArray{
  int n;string s;
  vector<int>sa,rank,lcp; vector<vector<int>>t; vector<int>lg;
  SuffixArray(){}
  SuffixArray(string _s){
    n=_s.size();s=_s;sa=suffix_array(s);
    rank.resize(n);
    for(int i=0;i<n;i++) rank[sa[i]]=i;
    costruct_lcp();prec();build();
  }
  void costruct_lcp(){
    int k=0;lcp.resize(n-1,0);
    for(int i=0;i<n;i++){
      if(rank[i]==n-1){k=0;continue;}
      int j=sa[rank[i]+1];
      while(i+k<n&&j+k<n&&s[i+k]==s[j+k]) k++;
      lcp[rank[i]]=k;
      if(k) k--; }
  }
  void prec(){lg.resize(n,0);for(int i=2;i<n;i++) lg[i]=lg[i/2]+1;}
  void build(){
    int sz=n-1;t.resize(sz);
    for(int i=0;i<sz;i++){t[i].resize(LG);t[i][0]=lcp[i];}
    for(int k=1;k<LG;++k) for(int i=0;i+(1<<k)-1<sz;++i) t[i][k]=min(t[i][k-1],t[i+(1<<(k-1))][k-1]);
  }
  int query(int l,int r){
    int k=lg[r-l+1];
    return min(t[l][k],t[r-(1<<k)+1][k]);
  }
  int get_lcp(int i,int j){
    if(i==j) return n-i;
    int l=rank[i],r=rank[j];
    if(l>r) swap(l,r);
    return query(l,r-1);
  }
  int lower_bound(string& t){
    int l=0,r=n-1,k=t.size(),ans=n;
    while(l<=r){
      int mid=l+r>>1;
      if(s.substr(sa[mid],min(n-sa[mid],k))>=t) ans=mid,r=mid-1;
      else l=mid+1; }
    return ans;
  }
  int upper_bound(string& t){
    int l=0,r=n-1,k=t.size(),ans=n;
    while(l<=r){
      int mid=l+r>>1;
      if(s.substr(sa[mid],min(n-sa[mid],k))>t) ans=mid,r=mid-1;
      else l=mid+1; }
    return ans;
  }
  pair<int,int> find_occurrence(int p,int len){
    p=rank[p];pair<int,int>ans={p,p};
    int l=0,r=p-1;
    while(l<=r){
      int mid=l+r>>1;
      if(query(mid,p-1)>=len) ans.first=mid,r=mid-1;
      else l=mid+1;}
    l=p+1;r=n-1;
    while(l<=r){
      int mid=l+r>>1;
      if(query(p,mid-1)>=len) ans.second=mid,l=mid+1;
      else r=mid-1; }
    return ans;
  }
};
void solve(){
  int n,m,q;cin>>n>>m>>q;
  string a,b;cin>>a>>b;
  string s=b+'@'+a+'#';
  SuffixArray sfa(s);
  vector<int>maxSub(m+1,0);
  for(int i=1;i<s.size();i++){
    if(sfa.sa[i]<m){
      int nw1=sfa.lcp[i-1],nw2=inf;
      if(sfa.sa[i-1]<m) nw2=maxSub[sfa.sa[i-1]+1];
      maxSub[sfa.sa[i]+1]=min(nw1,nw2);} }
  for(int i=s.size()-2;i>=0;i--){
    if(sfa.sa[i]<m){
      int nw1=sfa.lcp[i],nw2=inf;
      if(sfa.sa[i+1]<m) nw2=maxSub[sfa.sa[i+1]+1];
      maxSub[sfa.sa[i]+1]=max(maxSub[sfa.sa[i]+1],min(nw1,nw2)); }
  } build(m,maxSub);
  while(q--){
    int l,r;cin>>l>>r;
    int lo=1,hi=r-l+1,mid,ans=0;
    while(lo<=hi){
      mid=(lo+hi)/2;
      int nw=query(l,r-mid+1);
      if(nw>=mid) ans=mid,lo=mid+1;
      else hi=mid-1; }
    cout<<ans<<nl; } }
// ekhane lcp 1 len kom thake and lcp[i] means at sorted array, at index i + 1, it has a common prefix of len lcp[i]
\end{lstlisting}

\subsection{Trie}
\begin{lstlisting}
struct trienode{
   bool endmark;
   vector<int> child;
   trienode(int sz){
      endmark = false;
      child.resize(sz, 0);
   }
};
struct trie{
   int sz; char fst;
   vector<trienode> nodes;
   trie(int alpha_sz, char alpha_start){
      sz = alpha_sz;
      fst = alpha_start;
      // root at idx 0
      nodes.push_back(trienode(sz));
   }
   void insert(string& s){
      int cur = 0;
      for (char c : s){
         if (!nodes[cur].child[c - fst]){
            nodes[cur].child[c - fst] = nodes.size();
            nodes.push_back(trienode(sz));
         }
         cur = nodes[cur].child[c - fst];
      }
      nodes[cur].endmark = true;
   }
   bool search(string& s){
      int cur = 0;
      for (char c : s){
         if (!nodes[cur].child[c - fst]) return false;
         cur = nodes[cur].child[c - fst];
      }
      return nodes[cur].endmark; 
   }

   bool erase(string& s){
      int cur = 0;
      for (char c : s){
         if (!nodes[cur].child[c - fst])return false;
         cur = nodes[cur].child[c - fst];
      }
      if (!nodes[cur].endmark) return false;
      nodes[cur].endmark = false;
      return true;
   }
};
\end{lstlisting}

\subsection{BitTrie}
\begin{lstlisting}
const int Log = 31; 
struct BitTrie{
   int cur;     vector<array<int,2>> nxt;
   BitTrie(int n){
      cur=1;
      nxt=vector<array<int,2>>((n+5)*Log,{-1,-1});
   }
   void insert(int x){
      int node=0;
      for(int i=Log-1;i>=0;i--){
         int bt=((x>>i)&1);
         if(nxt[node][bt]==-1) nxt[node][bt]=cur++;
         node=nxt[node][bt];
      }
   }
   int bxor(int x){
      int node=0,res=0;
      for(int i=Log-1;i>=0;i--){
         int bt=(((x>>i)&1)^1);
         if(nxt[node][bt]!=-1){
            res|=(1LL<<i);
            node=nxt[node][bt];
         }else if(nxt[node][bt^1]!=-1){
            node=nxt[node][bt^1];
         }else break;
      }
      return res;
   }
};
\end{lstlisting}

\subsection{String Hashing}
\begin{lstlisting}
#define int int64_t
const int SZ=int(1e6)+9;
const int B1=151,B2=239;
const int mod1=127657753,mod2=987654319;
vector<int> P1(SZ,1),P2(SZ,1);
void pow_cal(){ // call pow_cal() inside main function once
   for(int i=1;i<SZ;i++){
      P1[i]=(P1[i-1]*B1)%mod1;
      P2[i]=(P2[i-1]*B2)%mod2;
   }
}
class Hash{
public:
   int n;   vector<int> H1,H2;
   Hash(const string& s){
      n=s.size();
      H1.assign(n+1,0); H2.assign(n+1,0);
      for(int i=1;i<=n;i++){
         H1[i]=(H1[i-1]+((s[i-1]-'a'+1)*P1[i])%mod1)%mod1;
         H2[i]=(H2[i-1]+((s[i-1]-'a'+1)*P2[i])%mod2)%mod2;
      }
   }
   pair<int,int> geth(int l, int r){
      int h1=(((H1[r]-H1[l-1]+mod1)%mod1)*P1[SZ-l])%mod1;
      int h2=(((H2[r]-H2[l-1]+mod2)%mod2)*P2[SZ-l])%mod2;
      return make_pair(h1,h2);
   }
   pair<int,int> geth()    return geth(1,n);
};
\end{lstlisting}

\subsection{2D Hashing}
\begin{lstlisting}
struct Hashing {
   int n,m;     static const int PX=3731,PY=2999,mod=998244353;
   vector<int> PWX,PWY;   vector<vector<int>> hs;
   Hashing(){}  
   Hashing(vector<string>& s){
      n=(int)s.size(), m=(int)s[0].size();
      hs.assign(n+1,vector<int>(m+1,0));
      PWX.assign(n+1,1);     PWY.assign(m+1,1);
      for(int i=0;i<n;i++) PWX[i+1]=1LL*PWX[i]*PX%mod;
      for(int i=0;i<m;i++) PWY[i+1]=1LL*PWY[i]*PY%mod;
      for(int i=0;i<n;i++){
         for(int j=0;j<m;j++){
            hs[i+1][j+1]=s[i][j]-'a'+1;
         }
      }
      for(int i=0;i<=n;i++){
         for(int j=0;j<m;j++){
         hs[i][j+1]=(hs[i][j+1]+1LL*hs[i][j]*PY%mod)%mod;
         }
      }
      for(int i=0;i<n;i++){
         for(int j=0;j<=m;j++){
         hs[i+1][j]=(hs[i+1][j]+1LL*hs[i][j]*PX%mod)%mod;
         }
      }
  }
  int get_hash(int x1,int y1,int x2,int y2) { // 1-indexed
    assert(1<=x1 && x1<=x2 && x2<=n);
    assert(1<=y1 && y1<=y2 && y2<=m);
    x1--;   y1--;
    int dx =x2-x1, dy =y2-y1;
    return (1LL * (hs[x2][y2] - 1LL * hs[x2][y1] * PWY[dy] % mod + mod) % mod - 1LL * (hs[x1][y2] - 1LL * hs[x1][y1] * PWY[dy] % mod + mod) % mod * PWX[dx] % mod + mod) % mod;
  }
  int get_hash()     return get_hash(1, 1, n, m);
};
\end{lstlisting}

\subsection{Longest Palindrome from any Index}
\begin{lstlisting}
// needs Hashing 
void lp(string& s){
   int n=s.size();      string t=s;
   reverse(t.begin(),t.end());
   Hash sh(s),th(t);
   auto f=[&](int L, int R){
      auto ss=sh.geth(L,R);
      auto tt=th.geth(n-R+1,n-L+1);
      return (ss==tt);
   };
   vector<int> R(n,1),L(n,1); 
   for(int i=n-1,r=n-1;i>=0;i--){
      if(r<n-1 && s[i]==s[r+1]){
         r++;
      }else{
         while(!f(i+1,r+1))      r--;
      }
      R[i]=max(R[i],r-i+1);
   } // R -> from index i to i+R[i]-1 is Pal 
   for(int i=0,l=0;i<n;i++){
      if(l>0 && s[i]==s[l-1]){
         l--;
      }else{
         while(!f(l+1,i+1))      l++;
      }
      L[i]=max(L[i],i-l+1);
   } // L-> from index i-L[i]+1 to i is Pal
}
\end{lstlisting}
\subsection{KMP}
\begin{lstlisting}
// returns the longest proper prefix array of pattern p
// lps[i] = longest proper prefix which is also suffix of p[0..i]
vector<int> build_lps(string p){
    int sz=p.size();
    vector<int>lps(sz+1,0);
    int j=0;
    lps[0]=0;
    for(int i=1;i<sz;i++){
        while(j>=0 && p[i]!=p[j]){
            if(j>=1) j=lps[j-1];
            else j=-1;
        }
        j++; lps[i]=j;
    }
    return lps;
}
vector<int> ans; // returns matches in vector ans (0-indexed)
void kmp(vector<int> lps,string s,string p){
    int psz=p.size(), sz=s.size(), j=0;
    for(int i=0;i<sz;i++){
        while(j>=0 && p[j]!=s[i]){
            if(j>=1) j=lps[j-1];
            else j=-1;
        }
        j++;
        if(j==psz){
            j=lps[j-1];
            // pattern found in string s at position i-psz+1
            ans.push_back(i-psz+1);
        }
        // after each loop, j = longest common suffix of s[0..i] which is also prefix of p
    }
}

\end{lstlisting}
\subsection{Min Lexo Rotation}
\begin{lstlisting}
int mnlex(string& s){
   int n=s.size(),sz=2*n,l=0,j=1,k=0;
   while(l+k<sz && j+k<sz){
      int a=((l+k>=n) ? s[l+k-n]: s[l+k]);
      int b=((j+k>=n) ? s[j+k-n]: s[j+k]);
      if(a==b){k++;}
      else if(a>b){ // if maxlex then (a<b)
         l=l+k+1; k=0;
         if(l<=j){ l=j+1; }
      }else{
         j=j+k+1; k=0;
         if(j<=l){ j=l+1; }
      }
   }
   return min(l,j);
}
\end{lstlisting}
\subsection{Aho Cora}
\begin{lstlisting}
// Aho-Corasick for multiple pattern search in a text
vector<map<char,int>> nxtCh; // Trie transitions
vector<int> fail,nxtWord,wordId; // Fail function, next word, word IDs
int addNode(){
    nxtCh.emplace_back();
    fail.push_back(0);
    nxtWord.push_back(0);
    wordId.push_back(-1);
    return wordId.size()-1; // return new node index
}
int insert(const string &s,int idx){
    int cur=0;
    for(char c:s){
        if(!nxtCh[cur].count(c)){
            int to=addNode(); // create node if missing
            nxtCh[cur][c]=to;
        }
        cur=nxtCh[cur][c]; // move to next node
    }
    if(wordId[cur]==-1)wordId[cur]=idx; // assign word index
    return wordId[cur];
}
void computeFailure(){
    queue<int> q;
    for(auto &it:nxtCh[0]) q.push(it.second);
    while(!q.empty()){
        int u=q.front();q.pop();
        for(auto [c,v]:nxtCh[u]){
            int f=fail[u]; // fail value for current node
            while(nxtCh[f].find(c)==nxtCh[f].end()&&f!=0) f=fail[f];
            fail[v]=nxtCh[f][c]; // fail for next node
            nxtWord[v]=wordId[fail[v]]!=-1?fail[v]:nxtWord[fail[v]];
            q.push(v);
        }
    }
}
// return occurrences of all patterns in text
vector<vector<int>> match(const string &s,int q){
    vector<vector<int>> ret(q);
    int cur=0;
    for(int i=0;i<s.size();i++){
        char c=s[i];
        while(nxtCh[cur].find(c)==nxtCh[cur].end()&&cur!=0) cur=fail[cur];
        cur=nxtCh[cur][c]; int u=cur;
        while(u!=0){
            if(wordId[u]!=-1) ret[wordId[u]].push_back(i);
            u=nxtWord[u];
        }
    }
    return ret;
}
int32_t main(){
    string text; cin>>text;
    int n; cin>>n;
    vector<int> id;
    addNode(); // root node
    for(int i=0;i<n;i++){
        string s; cin>>s;
        id.push_back(insert(s,i));
    }
    computeFailure();
    auto matching=match(text,n);
    for(int i=0;i<n;i++) cout<<"YN"[matching[id[i]].empty()]<<endl;
    return 0;
}

\end{lstlisting}
\section{Data Structures}

\subsection{2D Prefix Sum}
\begin{lstlisting}
const int N =1005;
  int a[N][N], pref[N][N];
  for (int i =1; i <=n; i++) {
    for (int j =1; j <=m; j++) {
      pref[i][j] =pref[i-1][j] + pref[i][j-1] - pref[i-1][j-1] + a[i][j];
    }
  }
  auto query=[&](int x1, int y1, int x2, int y2){
    int ans =pref[x2][y2] - pref[x1-1][y2] - pref[x2][y1-1] + pref[x1-1][y1-1];
  };
\end{lstlisting}

\subsection{DSU}
\begin{lstlisting}
struct DSU {
   vector<int> par,sz;     DSU(){}
   DSU(int n){ init(n); }
   void init(int n){
      par.resize(n);
      iota(par.begin(),par.end(),0);
      sz.assign(n,1);
   }
   int find(int v){
      while(v!=par[v]){
         v=par[v]=par[par[v]]; 
      }
      return v;
   }
   bool merge(int a,int b){
      a=find(a);        b=find(b);
      if(a==b) return false;
      if(sz[a]<sz[b]) swap(a, b);
      par[b]=a;         sz[a]+=sz[b];
      return true;
   }
};
\end{lstlisting}
\subsection{DSU on Trees}
\begin{lstlisting}
const int N = 1e5 + 9;
vector<int> g[N];
int ans[N], col[N], sz[N], cnt[N];
bool big[N];
void dfs(int u, int p) {
  sz[u] = 1;
  for (auto v : g[u]) {
    if (v == p) continue;
    dfs(v, u);
    sz[u] += sz[v];
  }
}
void add(int u, int p, int x) {
  cnt[col[u]] += x;
  for (auto v : g[u]) {
    if (v == p || big[v] == 1) continue;
    add(v, u, x);
  }
}
void dsu(int u, int p, bool keep) {
  int bigchild = -1, mx = -1;
  for (auto v : g[u]) {
    if (v == p) continue;
    if (sz[v] > mx) mx = sz[v], bigchild = v;
  }
  for (auto v : g[u]) {
    if (v == p || v == bigchild) continue;
    dsu(v, u, 0);
  }
  if (bigchild != -1) dsu(bigchild, u, 1), big[bigchild] = 1;
  add(u, p, 1);
  ans[u] = cnt[u];
  if (bigchild != -1) big[bigchild] = 0;
  if (keep == 0) add(u, p, -1);
}
\end{lstlisting}

\subsection{SegTree}
\begin{lstlisting}
class segTree{ // 1-base Indexing
public:
   int sz;  vector<int> t;    segTree(){}
   segTree(int l){   sz=l;
      if(__builtin_popcount(sz)!=1){
         int x=31-__builtin_clz(sz);
         sz=(1LL<<(x+1));
      }
      t.assign(((sz<<1)|1),0);
   }
   int combine(int& left, int& right){
      int tm=left+right;      return tm;
   }
   void update(int root, int tl, int tr, int pos, int x){
      if(tl==tr){ t[root]=x;     return;  }
      int mid=(tl+tr)/2;
      if(pos<=mid)    update((root<<1),tl,mid,pos,x);
      else            update(((root<<1)|1),mid+1,tr,pos,x);
      t[root]=combine(t[(root<<1)],t[((root<<1)|1)]);
   }
   int query(int root,  int tl, int tr, int l, int r){
      if(l>tr || r<tl)    return 0;
      if(l<=tl && r>=tr)  return t[root];
      int mid=(tl+tr)/2;
      auto q1=query((root<<1),tl,mid,l,r);
      auto q2=query(((root<<1)|1),mid+1,tr,l,r);
      return combine(q1,q2);
   }
   void update(int pos, int x)   update(1,0,sz-1,pos,x);  
   int query(int l, int r)       return query(1,0,sz-1,l,r);
};
\end{lstlisting}

\subsection{2D SegTree}
\begin{lstlisting}
// 2D Segment Tree for POINT update + RECTANGLE query
// Coordinates should be 1-based
template<typename T>
class SegTree2D {
private:
    int n, m;                  // grid size
    vector<vector<T>> tree;    // 4*n * 4*m
    // CHANGE THIS to modify behavior (sum / min / max / xor / gcd)
    T combine(const T &a, const T &b) return a + b;          // default: sum
    T neutral() return 0;              // identity element for combine
    void build_y(int vx, int lx, int rx, int vy, int ly, int ry) {
        if (ly == ry) {
            if (lx == rx) tree[vx][vy] = neutral();
            else tree[vx][vy] = combine(tree[vx*2][vy], tree[vx*2+1][vy]);
        } else {
            int my = (ly + ry) / 2;
            build_y(vx, lx, rx, vy*2, ly, my);
            build_y(vx, lx, rx, vy*2+1, my+1, ry);
            tree[vx][vy] = combine(tree[vx][vy*2], tree[vx][vy*2+1]);
        }
    }
    void build_x(int vx, int lx, int rx) {
        if (lx != rx) {
            int mx = (lx + rx) / 2;
            build_x(vx*2, lx, mx);
            build_x(vx*2+1, mx+1, rx);
        }
        build_y(vx, lx, rx, 1, 1, m);
    }
    // Update point (x,y) += val
    void update_y(int vx, int vy, int ly, int ry, int y, T val) {
        if (ly == ry) {
            tree[vx][vy] = combine(tree[vx][vy], val);
        } else {
            int my = (ly + ry) / 2;
            if (y <= my) update_y(vx, vy*2, ly, my, y, val);
            else update_y(vx, vy*2+1, my+1, ry, y, val);
            tree[vx][vy] = combine(tree[vx][vy*2], tree[vx][vy*2+1]);
        }
    }
    void update_x(int vx, int lx, int rx, int x, int y, T val) {
        if (lx == rx) {
            update_y(vx, 1, 1, m, y, val);
        } else {
            int mx = (lx + rx) / 2;
            if (x <= mx) update_x(vx*2, lx, mx, x, y, val);
            else update_x(vx*2+1, mx+1, rx, x, y, val);
            update_y(vx, 1, 1, m, y, val);
        }
    }
    // Query rectangle [x1..x2] × [y1..y2]
    T query_y(int vx, int vy, int ly, int ry, int y1, int y2) {
        if (y1 > y2) return neutral();
        if (ly == y1 && ry == y2) return tree[vx][vy];
        int my = (ly + ry) / 2;
        return combine(
            query_y(vx, vy*2, ly, my, y1, min(y2, my)),
            query_y(vx, vy*2+1, my+1, ry, max(y1, my+1), y2)
        );
    }
    T query_x(int vx, int lx, int rx, int x1, int x2, int y1, int y2) {
        if (x1 > x2) return neutral();
        if (lx == x1 && rx == x2) return query_y(vx, 1, 1, m, y1, y2);
        int mx = (lx + rx) / 2;
        return combine(
            query_x(vx*2, lx, mx, x1, min(x2, mx), y1, y2),
            query_x(vx*2+1, mx+1, rx, max(x1, mx+1), x2, y1, y2)
        );
    }
public:
    SegTree2D(int _n, int _m) : n(_n), m(_m) {
        tree.assign(4*n+10, vector<T>(4*m+10, neutral()));
        build_x(1, 1, n);
    }
    // add val to position (x,y)
    void add(int x, int y, T val) {
        update_x(1, 1, n, x, y, val);
    }
    // sum over rectangle
    T get(int x1, int y1, int x2, int y2) {
        return query_x(1, 1, n, x1, x2, y1, y2);
    }
};
\end{lstlisting}

\subsection{LazySeg}
\begin{lstlisting}
class segTree{
public:
   int sz;
   vector<int> t,lazy;
   segTree(){}
   segTree(int l){
      sz=l;
      if(__builtin_popcount(sz)!=1){
         int x=31-__builtin_clz(sz);
         sz=(1LL<<(x+1));
      }
      t.assign(2*sz+1,0);
      lazy.assign(2*sz+1,0);
   }
   void apply(int root, int tl, int tr){
      if(!lazy[root])     return;
      t[root]+=lazy[root];
      if(tl!=tr){
         lazy[root*2]+=lazy[root];
         lazy[root*2+1]+=lazy[root];
      }
      lazy[root]=0;
   }
   int combine(int& left, int& right){
      int tm=left+right;
      return tm;
   }
   void update(int root, int tl, int tr, int l, int r, int x){
      apply(root,tl,tr);
      if(l>tr || r<tl)        return;
      if(l<=tl && r>=tr){
         lazy[root]+=x;
         apply(root,tl,tr);
         return;
      }
      int mid=(tl+tr)/2;
      update(root*2,tl,mid,l,r,x);
      update(root*2+1,mid+1,tr,l,r,x);
      t[root]=combine(t[root*2],t[root*2+1]);
   }
   int query(int root,  int tl, int tr, int l, int r){
      apply(root,tl,tr);
      if(l>tr || r<tl)    return 0;
      if(l<=tl && r>=tr)  return t[root];
      int mid=(tl+tr)/2;
      auto q1=query(root*2,tl,mid,l,r);
      auto q2=query(root*2+1,mid+1,tr,l,r);
      return combine(q1,q2);
   }
   void update(int l, int r, int x){
      update(1,0,sz-1,l,r,x);
   }   
   int query(int l, int r){
      return query(1,0,sz-1,l,r);
   }
};
\end{lstlisting}

\subsection{Persistent Seg}
\begin{lstlisting}
const int N=int(2e5)+3;
struct node{
   int l=0,r=0,val=0;
}t[N*40];
#define lc(x)  t[x].l
#define rc(x)  t[x].r
#define v(x)  t[x].val
int T=0;
int build(int tl, int tr){
   int cur=++T;
   if(tl==tr) return cur;
   int mid=(tl+tr)/2;
   lc(cur)=build(tl,mid);
   rc(cur)=build(mid+1,tr);
   return cur;
}
int update(int prv, int tl, int tr, int pos, int x){
   int cur=++T;
   t[cur]=t[prv];
   if(tl==tr){
      v(cur)=x;
      return cur;
   }
   int mid=(tl+tr)/2;
   if(pos<=mid){
      lc(cur)=update(lc(prv),tl,mid,pos,x);
   }else{
      rc(cur)=update(rc(prv),mid+1,tr,pos,x);
   }
   v(cur)=v(lc(cur))+v(rc(cur));
   return cur;
}
int query(int cur, int tl, int tr, int l, int r){
   if(l>tr || r<tl)     return 0;
   if(tl>=l && tr<=r)   return v(cur);
   int mid=(tl+tr)/2;
   return query(lc(cur),tl,mid,l,r)+query(rc(cur),mid+1,tr,l,r);
}

\end{lstlisting}

\subsection{BIT}
\begin{lstlisting}
struct BIT{ // 1-base query
   int sz;     vector<int> bit;
   BIT(int n){ sz=n;    bit.resize(sz+1,0);  }
   void update(int pos, int v){
      while(pos<=sz){
         bit[pos]+=v;
         pos=pos+(pos&(-pos));
      }
   }
   int sum(int pos){    int s=0;
      while(pos>0){
         s+=bit[pos];
         pos=pos-(pos&(-pos));
      }     
      return s;
   }
   int query(int l, int r)    return sum(r)-sum(l-1);
};

\end{lstlisting}

\subsection{BIT Range Update}
\begin{lstlisting}
const int N = 3e5 + 9; // 1-base query
struct RangeBIT {
  int64_t M[N], A[N];
  RangeBIT() {
    memset(M, 0, sizeof(M));
    memset(A, 0, sizeof(A));
  }
  void update(int i, int64_t mul, int64_t add) {
    for (; i < N; i +=i & (-i)) {
      M[i] +=mul;
      A[i] +=add;
    }
  }
  void upd(int l, int r, int64_t x) {
    update(l, x, -x * (l - 1));
    update(r + 1, -x, x * r);
  }
  int64_t query(int i) {
    int64_t mul =0, add =0;
    for (; i > 0; i -=i & (-i)) {
      mul +=M[i];
      add +=A[i];
    }
    return (mul * i + add);
  }
  int64_t query(int l, int r) {
    return query(r) - query(l - 1);
  }
};
\end{lstlisting}

\subsection{2D BIT with Range Update}
\begin{lstlisting}
template<typename T> // BIT2D<int32_t> bt(N,M);
struct BIT2D{ // 1-base
   int n,m;
   vector<vector<T>> t;
   BIT2D(int n,int m):n(n),m(m),t(n+2,vector<T>(m+2,0)){}
   void update(int x,int y,T val){
      for(int i=x;i<=n;i+=(i&-i))for(int j=y;j<=m;j+=(j&-j))t[i][j]+=val;
   }
   void range_update(int x1,int y1,int x2,int y2,T val){
      update(x1,y1,val);    update(x1,y2+1,-val);
      update(x2+1,y1,-val);     update(x2+1,y2+1,val);
   }
   T query(int x,int y)const{
      T sum=0;
      for(int i=x;i>0;i-=(i&-i))for(int j=y;j>0;j-=(j&-j))sum+=t[i][j];
      return sum;
   }
   T query(int x1,int y1,int x2,int y2)const{
      return query(x2,y2)-query(x1-1,y2)-query(x2,y1-1)+query(x1-1,y1-1);
   }
   void reset(){
      for(int i=0;i<=n;++i)fill(t[i].begin(),t[i].end(),0);
   }
};
\end{lstlisting}

\subsection{Sparse Table}
\begin{lstlisting}
const int N =1e5 + 9;
int t[N][18], a[N];
void build(int n) { //RMQ
  for(int i =1; i <=n; ++i) t[i][0] =a[i];
  for(int k =1; k < 18; ++k) {
    for(int i =1; i + (1 << k) - 1 <=n; ++i) {
      t[i][k] =min(t[i][k - 1], t[i + (1 << (k - 1))][k - 1]);
    }
  }
}
int query(int l, int r) {
  int k =31 - __builtin_clz(r - l + 1);
  return min(t[l][k], t[r - (1 << k) + 1][k]);
} 
\end{lstlisting}

\subsection{Wavelet Tree}
\begin{lstlisting}
const int MAXN=3e5+9;
const int MAXV=1e9+9; // max value in array, can be negative
struct wavelet_tree{
    int lo,hi;
    wavelet_tree *l,*r;
    int *b,*c,bsz,csz; // c holds prefix sums
    wavelet_tree(){lo=1,hi=0,bsz=0,csz=0,l=NULL,r=NULL;}
    void init(int *from,int *to,int x,int y){
        lo=x,hi=y;
        if(from>=to)return;
        int mid=(lo+hi)>>1;
        auto f=[mid](int v){return v<=mid;};
        b=(int*)malloc((to-from+2)*sizeof(int));bsz=0;b[bsz++]=0;
        c=(int*)malloc((to-from+2)*sizeof(int));csz=0;c[csz++]=0;
        for(auto it=from;it!=to;it++){
            b[bsz]=b[bsz-1]+f(*it);
            c[csz]=c[csz-1]+(*it);
            bsz++;csz++;
        }
        if(hi==lo)return;
        auto pivot=stable_partition(from,to,f);
        l=new wavelet_tree();l->init(from,pivot,lo,mid);
        r=new wavelet_tree();r->init(pivot,to,mid+1,hi);
    }
    // kth smallest element in [l,r]
    int kth(int l,int r,int k){
        if(l>r)return 0;
        if(lo==hi)return lo;
        int inLeft=b[r]-b[l-1],lb=b[l-1],rb=b[r];
        if(k<=inLeft)return this->l->kth(lb+1,rb,k);
        return this->r->kth(l-lb,r-rb,k-inLeft);
    }
    // count of numbers <= k in [l,r]
    int LTE(int l,int r,int k){
        if(l>r||k<lo)return 0;
        if(hi<=k)return r-l+1;
        int lb=b[l-1],rb=b[r];
        return this->l->LTE(lb+1,rb,k)+this->r->LTE(l-lb,r-rb,k);
    }
    // count occurrence of k in [l,r]
    int count(int l,int r,int k){
        if(l>r||k<lo||k>hi)return 0;
        if(lo==hi)return r-l+1;
        int lb=b[l-1],rb=b[r],mid=(lo+hi)>>1;
        if(k<=mid)return this->l->count(lb+1,rb,k);
        return this->r->count(l-lb,r-rb,k);
    }
    // sum of numbers <= k in [l,r]
    int sum(int l,int r,int k){
        if(l>r||k<lo)return 0;
        if(hi<=k)return c[r]-c[l-1];
        int lb=b[l-1],rb=b[r];
        return this->l->sum(lb+1,rb,k)+this->r->sum(l-lb,r-rb,k);
    }
    ~wavelet_tree(){delete l;delete r;}
};
wavelet_tree t;
int a[MAXN];
int main(){
    int i,n,q,l,r,k,x;
    cin>>n;
    for(i=1;i<=n;i++)cin>>a[i];
    t.init(a+1,a+n+1,-MAXV,MAXV);
    // beware! after init(), a[] is not same
    cin>>q;
    while(q--){
        cin>>x>>l>>r>>k;
        if(x==0)cout<<t.kth(l,r,k)<<endl; // kth smallest
        else if(x==1)cout<<t.LTE(l,r,k)<<endl; // <= k
        else if(x==2)cout<<t.count(l,r,k)<<endl; // count k
        else if(x==3)cout<<t.sum(l,r,k)<<endl; // sum <= k
    }
    return 0;
}
\end{lstlisting}
\subsection{MOs Algo}
\begin{lstlisting}
const int N = 1e6 + 9, B = 440;

struct query {
  int l, r, id;
  bool operator < (const query &x) const {
    if(l / B == x.l / B) return ((l / B) & 1) ? r > x.r : r < x.r;
    return l / B < x.l / B;
  }
} Q[N];
int cnt[N], a[N];
long long sum;
inline void add_left(int i) {
  int x = a[i];
  sum += 1LL * (cnt[x] + cnt[x] + 1) * x;
  ++cnt[x];
}
inline void add_right(int i) {
  int x = a[i];
  sum += 1LL * (cnt[x] + cnt[x] + 1) * x;
  ++cnt[x];
}
inline void rem_left(int i) {
  int x = a[i];
  sum -= 1LL * (cnt[x] + cnt[x] - 1) * x;
  --cnt[x];
}
inline void rem_right(int i) {
  int x = a[i];
  sum -= 1LL * (cnt[x] + cnt[x] - 1) * x;
  --cnt[x];
}
long long ans[N];
\end{lstlisting}

\subsection{SegTree Hashing}
\begin{lstlisting}
// call pow_cal() inside Main Function
const int SZ=int(1e6)+9;
const int B1=151,mod1=127657753;
const int B2=239,mod2=987654319;
vector<int> P1(SZ,1),P2(SZ,1);
void pow_cal(){
   for(int i=1;i<SZ;i++){
      P1[i]=(P1[i-1]*B1)%mod1;
      P2[i]=(P2[i-1]*B2)%mod2;
   }
}
class segHash{
public:
   int n;   vector<int> T1,T2;
   segHash(const string& s){
      n=s.size();
      T1.assign(n*4+1,0);
      T2.assign(n*4+1,0);
      build(s,1,0,n-1);
   }
   void build(const string& s, int v, int tl, int tr){
      if(tl==tr){
         T1[v]=(s[tl]-'a'+1)%mod1;
         T2[v]=(s[tl]-'a'+1)%mod2;
         return;
      }
      int mid=(tl+tr)/2;
      build(s,v*2,tl,mid);
      build(s,v*2+1,mid+1,tr);
      T1[v]=(((T1[v*2]*P1[tr-mid])%mod1)+T1[v*2+1])%mod1;
      T2[v]=(((T2[v*2]*P2[tr-mid])%mod2)+T2[v*2+1])%mod2;
   }
   void update(int v, int tl, int tr, int pos, char c){
      if(tl==tr){
         T1[v]=(c-'a'+1)%mod1;
         T2[v]=(c-'a'+1)%mod2;
         return;
      }
      int mid=(tl+tr)/2;
      if(pos<=mid){
         update(v*2,tl,mid,pos,c);
      }else{
         update(v*2+1,mid+1,tr,pos,c);
      }
      T1[v]=(((T1[v*2]*P1[tr-mid])%mod1)+T1[v*2+1])%mod1;
      T2[v]=(((T2[v*2]*P2[tr-mid])%mod2)+T2[v*2+1])%mod2;
   }
   pair<int,int> query(int v, int tl, int tr, int l, int r){
      if(tl>r || tr<l)     return {0,0};
      if(l<=tl && tr<=r)   return {T1[v],T2[v]};
      int mid=(tl+tr)/2;
      auto left=query(v*2,tl,mid,l,min(mid,r));
      auto right=query(v*2+1,mid+1,tr,max(l,mid+1),r);
      int len=max(int(0),r-max(l,mid+1)+1);
      int H1=(((left.first*P1[len])%mod1)+right.first)%mod1;
      int H2=(((left.second*P2[len])%mod2)+right.second)%mod2;
      return {H1,H2};
   }
   void update(int pos, char c)  return update(1,0,n-1,pos-1,c);
   pair<int,int> geth(int l,int r)  return query(1,0,n-1,l-1,r-1);
};
\end{lstlisting}

\subsection{RMQ}
\begin{lstlisting}
struct RMQ {
    int n,t[2 * N];
    void build(int _n) {
        n =_n;
        for (int i =2 * n - 1; i >=n; i--) {
            t[i] =psum[i - n];
        }
        for (int i =n - 1; i > 0; --i) {
            t[i] =min(t[i << 1], t[i << 1 | 1]);
        }
    }
    int query(int l, int r) {
        r++;    int minV =1e18;
        for (l +=n, r +=n; l < r; l >>=1, r >>=1) {
            if (l & 1) minV =min(minV, t[l++]);
            if (r & 1) minV =min(minV, t[--r]);
        }
        return minV;
    }
} r;
\end{lstlisting}

\subsection{Merge Sort Tree}
\begin{lstlisting}
const int inf=1e9;
struct MergeSortTree{
    int n;
    vector<vector<int>>tree;
    MergeSortTree(const vector<int>&arr){
        n=arr.size();
        tree.resize(4*n);
        build(0,0,n-1,arr);
    }
    void build(int node,int start,int end,const vector<int>&arr){
        if(start==end){
            tree[node].push_back(arr[start]);
            return;
        }
        int mid=(start+end)/2;
        build(2*node+1,start,mid,arr);
        build(2*node+2,mid+1,end,arr);
        merge(tree[2*node+1].begin(),tree[2*node+1].end(),
              tree[2*node+2].begin(),tree[2*node+2].end(),
              back_inserter(tree[node]));
    }
    // lower bound of x in the range [l, r]
    int query(int node,int start,int end,int l,int r,int x){
        if(l>end||r<start)return inf;
        if(l<=start&&end<=r){
            auto it=lower_bound(tree[node].begin(),tree[node].end(),x);
            if(it!=tree[node].end())return *it;
            return inf;
        }
        int mid=(start+end)/2;
        int L=query(2*node+1,start,mid,l,r,x);
        int R=query(2*node+2,mid+1,end,l,r,x);
        return min(L,R);
    }
    // wrapper function for querying the tree
    int query(int l,int r,int x){
        return query(0,0,n-1,l,r,x);
    }
};
int main(){
    vector<int>arr={3,1,5,2,4};
    MergeSortTree mst(arr);
    cout<<mst.query(1,2,2)<<endl; // Output: 5 (lower bound of 2 in [1,2])
    cout<<mst.query(0,4,0)<<endl; // Output: 1 (lower bound of 0 in [0,4])
    cout<<mst.query(0,4,6)<<endl; // Output: inf (6 not present)
    return 0;
}
\end{lstlisting}

\subsection{Centroid Decomp}
\begin{lstlisting}
const int N = 1e5 + 9;
vector<int> g[N];
int sz[N];
void dfs(int u, int p = 0) {
	sz[u] = 1;
	for (auto v: g[u]) {
		if (v ^ p) {
			dfs(v, u);
			sz[u] += sz[v];
		}
	}
}
vector<int> get_centroids(int n) {
	int u = 1;
	dfs(u);
	while (1) {
		int tmp = -1;
		for (auto v: g[u]) {
			if (sz[v] > sz[u]) continue;
			if (2 * sz[v] >= n) tmp = v;
		}
		if (tmp == -1) break;
		u = tmp;
	}
	dfs(u);
	for (auto v: g[u]) {
		if (2 * sz[v] == n) {
			return {u, v};
		}
	}
	return {u};
}
\end{lstlisting}

\subsection{HLD}
\begin{lstlisting}
template<typename T>
struct segTree{
   int n;   vector<T> t, lazy;
   T neutral =0; // 0 for sum, -inf for max, +inf for min
   SegTree(int n =0): n(n), t(4*n, neutral), lazy(4*n, 0){}
   T combine(T a, T b){}
   void push(int v, int tl, int tr){}
   void build(const vector<T>& a, int v, int tl, int tr){}
   void build(const vector<T>& a) { build(a, 1, 1, n); }
   void update(int v, int tl, int tr, int l, int r, T add){}
   void update(int l, int r, T add){update(1, 1, n, l, r, add);}
   T query(int v, int tl, int tr, int l, int r){}
   T query(int l, int r) { return query(1, 1, n, l, r); }
   void set_value(int pos, T val) { // overwrite
      T cur =query(pos, pos);
      update(pos, pos, val - cur);
   }
};
template<typename T>
struct HLD {
   int n,timer;   vector<vector<int>> g;
   vector<int> parent, depth, heavy, head, pos, sz;
   vector<T> flat;   SegTree<T> seg;
   HLD(int n): n(n), g(n+1), parent(n+1), depth(n+1),heavy(n+1,-1), head(n+1), pos(n+1), sz(n+1), timer(0) {}
   void add_edge(int u, int v) {
      g[u].push_back(v);
      g[v].push_back(u);
   }
   int dfs(int u, int p) {
      parent[u] =p; sz[u] =1;
      int max_sz =0;
      for (int v : g[u]) if (v !=p) {
         depth[v] =depth[u] + 1;
         int sub =dfs(v, u);    sz[u] +=sub;
         if (sub > max_sz) max_sz =sub, heavy[u] =v;
      }
      return sz[u];
   }
   void decompose(int u, int h, const vector<T>& val) {
      head[u] =h;   pos[u] =++timer; flat[timer] =val[u];
      if (heavy[u] !=-1) decompose(heavy[u], h, val);
      for (int v : g[u])if (v !=parent[u] && v !=heavy[u])   decompose(v, v, val);
   }
   void init(const vector<T>& val, int root =1) {
      flat.assign(n+1, 0);
      dfs(root, 0);
      timer =0;
      decompose(root, root, val);
      seg =SegTree<T>(n);
      seg.build(flat);
   }
   void update_path(int u, int v, T add) {    // Path range add
      while (head[u] !=head[v]) {
         if (depth[head[u]] < depth[head[v]]) swap(u, v);
         seg.update(pos[head[u]], pos[u], add);
         u =parent[head[u]];
      }
      if (depth[u] > depth[v]) swap(u, v);
      seg.update(pos[u], pos[v], add);
   }
   T query_path(int u, int v) {    // Path query (need to update)
      T res =0; // check
      while (head[u] !=head[v]) {
         if (depth[head[u]] < depth[head[v]]) swap(u, v);
         res +=seg.query(pos[head[u]], pos[u]);
         u =parent[head[u]];
      }
      if (depth[u] > depth[v]) swap(u, v);
      res +=seg.query(pos[u], pos[v]);
      return res;
   }
   void update_subtree(int u, T add) {    // Subtree range add
      seg.update(pos[u], pos[u] + sz[u] - 1, add);
   }
   T query_subtree(int u) {   // Subtree query
      return seg.query(pos[u], pos[u] + sz[u] - 1);
   }
   void set_value(int u, T val) {   // Point set
      seg.set_value(pos[u], val);
   }
   int lca(int u, int v) {
      while (head[u] !=head[v]) {
         if (depth[head[u]] < depth[head[v]]) swap(u, v);
         u =parent[head[u]];
      }
      return depth[u] < depth[v] ? u : v;
   }
};
\end{lstlisting}

\section{Graph and Trees}

\subsection{LCA}
\begin{lstlisting}
const int N =3e5 + 9, LG =18;
vector<int> g[N];
int par[N][LG + 1], dep[N], sz[N];
void dfs(int u, int p =0) {
    par[u][0] =p;
    dep[u] =dep[p] + 1;
    sz[u] =1;
    for (int i =1; i <=LG; i++) par[u][i] =par[par[u][i - 1]][i - 1];
    for (auto v: g[u]) if (v !=p) {
        dfs(v, u);
        sz[u] +=sz[v];
    }
}
int lca(int u, int v) {
    if (dep[u] < dep[v]) swap(u, v);
    for (int k =LG; k >=0; k--) if (dep[par[u][k]] >=dep[v]) u =par[u][k];
    if (u ==v) return u;
    for (int k =LG; k >=0; k--) if (par[u][k] !=par[v][k]) u =par[u][k], v =par[v][k];
    return par[u][0];
}
int kth(int u, int k) {
    assert(k >=0);
    for (int i =0; i <=LG; i++) if (k & (1 << i)) u =par[u][i];
    return u;
}
int dist(int u, int v) {
    int l =lca(u, v);
    return dep[u] + dep[v] - (dep[l] << 1);
}
//kth node from u to v, 0th node is u
int go(int u, int v, int k) {
    int l =lca(u, v);
    int d =dep[u] + dep[v] - (dep[l] << 1);
    assert(k <=d);
    if (dep[l] + k <=dep[u]) return kth(u, k);
    k -=dep[u] - dep[l];
    return kth(v, dep[v] - dep[l] - k);
}
\end{lstlisting}

\subsection{Tree Diameter}
\begin{lstlisting}
auto fn(int s, vector<vector<int>>& g){
   vector<bool> vis(g.size()+1,false);
   queue<array<int,2>> q;
   q.push({s,0}); vis[s]=true;
   array<int,2> ans={s,0};
   while(!q.empty()){
      auto [x,y]=ans=q.front();
      q.pop();
      for(auto &i: g[x]){
         if(!vis[i]){
            vis[i]=true;
            q.push({i,y+1});
         }
      }
   }
   return ans;
}
\end{lstlisting}

\subsection{Cycle Detection}
\begin{lstlisting}
const int N =5e5 + 9;
vector<pair<int, int>> g[N];
int vis[N], par[N], e_id[N];
vector<int> cycle; // simple cycle, contains edge ids
bool dfs(int u) {
  if (!cycle.empty()) return 1;
  vis[u] =1;
  for (auto [v, id] : g[u]) {
    if (v !=par[u]) {
      if (vis[v] ==0) {
        par[v] =u;
        e_id[v] =id;
        if (dfs(v)) return 1;
      }else if (vis[v] ==1) {
        cycle.push_back(id);    // cycle here
        for (int x =u; x !=v; x =par[x]) {
          cycle.push_back(e_id[x]);
        }
        return 1;
      }
    }
  }
  vis[u] =2;
  return 0;
}
\end{lstlisting}

\subsection{SCC (Kosaraju)}
\begin{lstlisting}
// DFS Function to reach destination
bool dfs(int curr, int des, vector<vector<int>> &adj, vector<int> &vis) {
    if (curr ==des) return true;
    vis[curr] =1;
    for (auto x : adj[curr]) {
        if (!vis[x] && dfs(x, des, adj, vis)) return true;
    }
    return false;
}

// To tell whether there is path from source to destination
bool isPath(int src, int des, vector<vector<int>> &adj) {
    vector<int> vis(adj.size() + 1, 0);
    return dfs(src, des, adj, vis);
}

// Function to return all the strongly connected components of a graph
vector<vector<int>> findSCC(int n, vector<vector<int>> &a) {
    vector<vector<int>> ans;
    vector<int> is_scc(n + 1, 0);
    vector<vector<int>> adj(n + 1);
    for (int i =0; i < a.size(); i++) {
        adj[a[i][0]].push_back(a[i][1]);
    }
    for (int i =1; i <=n; i++) {
        if (!is_scc[i]) {
            vector<int> scc;
            scc.push_back(i);
            for (int j =i + 1; j <=n; j++) {
                if (!is_scc[j] && isPath(i, j, adj) && isPath(j, i, adj)) {
                    is_scc[j] =1;
                    scc.push_back(j);
                }
            }
            ans.push_back(scc);
        }
    }
    return ans;
}
\end{lstlisting}

\subsection{Articulation Point}
\begin{lstlisting}
const int N =3e5 + 9;
int T, low[N], dis[N], art[N];
vector<int> g[N];
void dfs(int u, int pre =0) {
  low[u] =dis[u] =++T;
  int child =0;
  for(auto v: g[u]) {
    if(!dis[v]) {
      dfs(v, u);
      low[u] =min(low[u], low[v]);
      if(low[v] >=dis[u] && pre !=0) art[u] =1;
      ++child;
    }
    else if(v !=pre) low[u] =min(low[u], dis[v]);
  }
  if(pre ==0 && child > 1) art[u] =1;
}
\end{lstlisting}

\subsection{Articulation Bridge TECC}
\begin{lstlisting}
struct TECC{
   using Edge=array<int,2>;
   int n,cnt_comp;
   vector<vector<int>> G,T;
   vector<bool> vis;
   vector<int> tin,low,comp;
   vector<Edge> bridges;
   void low_dfs(int u, int par, int& cur){
      vis[u]=true;
      tin[u]=cur++;
      low[u]=n+1;
      bool usedPar=false;
      for(auto v: G[u]){
         if(vis[v]){
            if(v!=par || usedPar){
               low[u]=min(low[u],tin[v]);
            }else{
               usedPar=true;
            }
            continue;
         }
         low_dfs(v,u,cur);
         low[u]=min(low[u],low[v]);
      }
   }
   void build_dfs(int u, int cid){
      comp[u]=cid;
      for(auto v: G[u]){
         if(comp[v]!=-1)      continue;
         if(tin[u]<low[v]){
            bridges.push_back({u,v});
            cnt_comp++;
            build_dfs(v,cnt_comp);
         }else{
            build_dfs(v,cid);
         }
      }
   }
   TECC(int _n,const vector<vector<int>>& g){
      n=_n; G=g; cnt_comp=1;
      tin.resize(n+1);low.resize(n+1);
      vis.assign(n+1,false);
      comp.assign(n+1,-1);
      int cur=1;
      for(int i=1;i<=n;i++){
         if(vis[i])     continue;
         low_dfs(i,-1,cur);
         build_dfs(i,cnt_comp);
         cnt_comp++;
      }
   }
   void build_tree(){
      T.assign(cnt_comp,{});
      for(auto &e: bridges){
         int u=comp[e[0]],v=comp[e[1]];
         T[u].push_back(v);
         T[v].push_back(u);
      }
   }
};
\end{lstlisting}

\subsection{Dijsktra}
\begin{lstlisting}
const int N =int(2e5)+3,inf=int(1e16);
vector<pair<int,int>> g[N]; // in graph weight first & neighbour second
vector<int> dis(N,inf);
void SP(int source){
   priority_queue<pair<int,int>,vector<pair<int,int>>,greater<pair<int,int>>> pq;
   dis[source]=0;          pq.push({0,source});
   par[source]=-1;        // if parent is necessary 
   while(!pq.empty()){
      int v=pq.top().second,wt=pq.top().first;  pq.pop();
      if(dis[v]<wt)    continue; 
      for(auto &it:g[v]){ 
         int child_v=it.second,child_wt=it.first;
         if(dis[v]+child_wt<dis[child_v]){
            par[child_v]=v;    // if parent is necessary 
            dis[child_v]=dis[v]+child_wt;
            pq.push({dis[child_v],child_v});
         }
      }
   }
}
\end{lstlisting}

\subsection{Bellman Ford}
\begin{lstlisting}
struct edge{ int a,b,cost; };
vector<edge> node;
int n,m,root_node;
pair<vector<int>, bool> bellman_ford(){
    vector<int> dis(n,inf);
    dis[root_node]=0;
    // relax all edges (n-1) times
    for(int i=0;i<n-1;i++){
        for(int j=0;j<m;j++){
            if(dis[node[j].a]<inf){
                dis[node[j].b]=min(dis[node[j].b],dis[node[j].a]+node[j].cost);
            }
        }
    }
    // check for negative cycle
    bool has_neg_cycle=false;
    for(int j=0;j<m;j++){
        if(dis[node[j].a]<inf &&
           dis[node[j].a]+node[j].cost<dis[node[j].b]){
            has_neg_cycle=true;
            break;
        }
    }
    return {dis,has_neg_cycle};
}
\end{lstlisting}

\subsection{Dinic's Algo}
\begin{lstlisting}
const int64_t inf = 1LL << 60; 
struct Dinic {
   struct Edge {
      int to, rev; int64_t flow, w; int id;
   };
   int n, s, t, mxid;vector<int> d, done;
   vector<int64_t> flow_through;vector<vector<Edge>> g;
   Dinic() {}
   Dinic(int _n) {
      n = _n + 10;
      mxid = 0;
      g.resize(n);
   }
   void add_edge(int u, int v, int64_t w, int id = -1) {
      Edge a = {v, (int)g[v].size(), 0, w, id};
      Edge b = {u, (int)g[u].size(), 0, 0, -2}; // cap(b) = w for bidirectional
      g[u].emplace_back(a);
      g[v].emplace_back(b);
      if (id >= 0) mxid = max(mxid, id);
   }
   bool bfs() {
      d.assign(n, -1); d[s] = 0;
      queue<int> q; q.push(s);
      while (!q.empty()) {
         int u = q.front();
         q.pop();
         for (auto &e : g[u]) {
            int v = e.to;
            if (d[v] == -1 && e.flow < e.w) {
               d[v] = d[u] + 1;
               q.push(v);
            }
         }
      }
      return d[t] != -1;
   }
   int64_t dfs(int u, int64_t flow) {
      if (u == t) return flow;
      for (int &i = done[u]; i < (int)g[u].size(); i++) {
         Edge &e = g[u][i];
         if (e.w <= e.flow) continue;
         int v = e.to;
         if (d[v] == d[u] + 1) {
            int64_t nw = dfs(v, min(flow, e.w - e.flow));
            if (nw > 0) {
               e.flow += nw;
               g[v][e.rev].flow -= nw;
               return nw;
            }
         }
      }
      return 0;
   }
   int64_t max_flow(int _s, int _t) {
      s = _s;
      t = _t;
      int64_t flow = 0;
      while (bfs()) {
         done.assign(n, 0);
         while (int64_t nw = dfs(s, inf)) {
            flow += nw;
         }
      }
      flow_through.assign(mxid+10,int64_t(0));
      for (int i = 0; i < n; i++) {
         for (auto &e : g[i]) {
            if (e.id >= 0) {
               flow_through[e.id] = e.flow;
            }
         }
      }
      return flow;
   }
};
\end{lstlisting}

\subsection{Min Cost Flow}
\begin{lstlisting}
const int64_t inf = int(1e18);
struct MCF{
   struct Edge {
      int to, rev; int64_t cap, cost, flow = 0;
   };
   int n; vector<vector<Edge>> g;
   vector<int64_t> h, dist; vector<int> prevv, preve;
   MCF(int _n = 0) {
      init(_n);
   }
   void init(int _n) {
      n = _n;
      g.assign(n, {});
      h.assign(n, 0);
      dist.assign(n, 0);
      prevv.assign(n, 0);
      preve.assign(n, 0);
   }
   void add(int u, int v, int64_t cap, int64_t cost) {
      g[u].push_back({v, (int)g[v].size(), cap, cost});
      g[v].push_back({u, (int)g[u].size() - 1, 0, -cost});
   }
   pair<int64_t, int64_t> flow(int s, int t, int64_t need = inf) {
      pair<int64_t, int64_t> res = {0, 0};
      fill(h.begin(), h.end(), 0LL);
      while (need > 0) {
         priority_queue<pair<int64_t, int>, vector<pair<int64_t, int>>, greater<pair<int64_t, int>>> pq;
         fill(dist.begin(), dist.end(), inf);
         dist[s] = 0;
         pq.emplace(0, s);
         while (!pq.empty()) {
            auto [d, v] = pq.top();
            pq.pop();
            if (dist[v] < d) continue;
            for (int i = 0; i < (int)g[v].size(); ++i) {
               Edge &e = g[v][i];
               if (e.cap > e.flow) {
                  int64_t nc = d + e.cost + h[v] - h[e.to];
                  if (dist[e.to] > nc) {
                     dist[e.to] = nc;
                     prevv[e.to] = v;
                     preve[e.to] = i;
                     pq.emplace(nc, e.to);
                  }
               }
            }
         }
         if (dist[t] == inf)  break;
         for (int i = 0; i < n; ++i) {
            if (dist[i] < inf) {
               h[i] += dist[i];
            }
         }
         int64_t push = need;
         for (int v = t; v != s; v = prevv[v]) {
            push = min(push, g[prevv[v]][preve[v]].cap - g[prevv[v]][preve[v]].flow);
         }
         need -= push;
         res.first += push;
         res.second += push * h[t];
         for (int v = t; v != s; v = prevv[v]) {
            Edge &e = g[prevv[v]][preve[v]];
            e.flow += push;
            g[e.to][e.rev].flow -= push;
         }
      }
      return res;
   }
   void clear_flow() {
      for (auto &vec : g) {
         for (auto &e : vec) {
            e.flow = 0;
         }
      }
   }
};
\end{lstlisting}

\section{DP}
\subsection{Basic}
\begin{lstlisting}
# 3. Digit DP
vector<int> A, B;
ll dp[20][150][2][2];
// dp[curPos][sum][a is smaller than R][a is bigger than L]
ll call(int pos, int sum, int small, int big){
   if (pos == (int)B.size()) return (ll)sum;
   ll &res = dp[pos][sum][small][big];
   if (~res) return res;
   res = 0;
   int start = A[pos], stop = B[pos];
   if (big)   start = 0;
   if (small) stop = 9;
   for (int dgt = start; dgt <= stop; dgt++)
      res += call(pos + 1, sum + dgt, small | dgt<B[pos], big | dgt> A[pos]);
   return res;
}
ll solve(ll a, ll b){
   A.clear();
   B.clear();
   while (b > 0){
      B.push_back(b % 10);
      b /= 10;
   }
   while (a > 0){
      A.push_back(a % 10);
      a /= 10;
   }
   while (A.size() < B.size())
      A.push_back(0);
   reverse(A.begin(), A.end());
   reverse(B.begin(), B.end());
   memset(dp, -1, sizeof(dp));
   return call(0, 0, 0, 0);
}
# 4. knapsack with high capacity; //O(N * sum profit)
ll dp[maxN * maxV + 5];
vector<ll> wt, profit;
for (i = 0; i < n; i++){
   w = wt[i]; v = profit[i]; sum += v;
   for (j = sum; j >= v; j--){
      dp[j] = min(dp[j], w + dp[j - v]);
      if (dp[j] <= cap)
         ans = max(ans, j);
   }
}
# 5. knapsack with bitset
bitset<MAX> possible;
possible.reset();
possible[0] = 1;
for (int coin : coins){
   auto new_poss = possible << coin;
   possible |= new_poss;
}
\end{lstlisting}

\subsection{Optimization}
\begin{lstlisting}
# 1. Aliens Trick
pair<ll, ll> solveWithPenalty(ll p){
   ll ans = 0, cnt = 0;
   // 1) add p with targets 2) cal ans 3) remove p
   return {ans, cnt};
}
ll Alien(ll l, ll r, ll need){
   ll ans = -1;
   while (l <= r){
      ll mid = l + (r - l) / 2;
      // solveWithPenalty() retrun {ansWithPenalty, cnt}
      pair<ll, ll> res = solveWithPenalty(mid);
      if (res.second >= need)
         ans = res.first - mid * need, l = mid + 1;
      else
         r = mid - 1;
   }
   return ans;
}
# 2. Divide and Conquer Optimization
ll dp[N];
ll cost(int l, int r) { return 1; } // dummy
void solve(int l, int r, int ql, int qr){
   if (l > r)
      return;
   int mid = l + (r - l) / 2;
   ll pos = 0, val = INT_MAX; // check min or max ?
   for (int k = ql; k <= min(mid, qr); k++){
      ll ret = cost(k, mid);
      if (ret < val)
         val = ret, pos = k; //min or max? 
   }
   dp[mid] = val;
   // for H[j]<=H[j+1] i.e. cost(1,j)<=cost(1,j+1)
   solve(l, mid - 1, ql, pos);
   solve(mid + 1, r, pos, qr);
   // for H[j]>=H[j+1] i.e. cost(1,j)>=cost(1,j+1)
   solve(l, mid - 1, pos, qr);
   solve(mid + 1, r, ql, pos); 
}
\end{lstlisting}

\subsection{Submask Enumeration}
\begin{lstlisting}
for(int k=1;k<(1<<n);k++){
   for(int sub=(k&(k-1));;sub=(k&(sub-1))){
      if(sub==0)  break;
   }
}
\end{lstlisting}

\subsection{SOS DP}
\begin{lstlisting}
const int B =20;
int a[1 << B], f[1 << B], g[1 << B];
// sum over subsets
for (int i =0; i < B; i++) {
    for (int mask =0; mask < (1 << B); mask++) {
        if ((mask & (1 << i)) !=0) {
            f[mask] +=f[mask ^ (1 << i)];
        }
    }
}
// sum over supersets
for (int i =0; i < B; i++) {
    for (int mask =(1 << B) - 1 ; mask >=0 ; mask--) {
        if ((mask & (1 << i)) ==0) g[mask] +=g[mask ^ (1 << i)] ;
    }
}
\end{lstlisting}
\subsection{CHT}
\begin{lstlisting}
const ll is_query=-(1LL<<62);
struct line{
    ll m,b;
    mutable function<const line*()>succ;
    bool operator<(const line&rhs)const{
        if(rhs.b!=is_query)return m<rhs.m;
        const line*s=succ();
        if(!s)return 0;
        ll x=rhs.m;
        return b-s->b<(s->m-m)*x;
    }
};
struct dynamic_hull:multiset<line>{
    const ll inf=LLONG_MAX;
    bool bad(iterator y){
        auto z=next(y);
        if(y==begin()){
            if(z==end())return 0;
            return y->m==z->m&&y->b<=z->b;
        }
        auto x=prev(y);
        if(z==end())return y->m==x->m&&y->b<=x->b;
        ll v1=(x->b-y->b);
        if(y->m==x->m)v1=x->b>y->b?inf:-inf;
        else v1/=(y->m-x->m);
        ll v2=(y->b-z->b);
        if(z->m==y->m)v2=y->b>z->b?inf:-inf;
        else v2/=(z->m-y->m);
        return v1>=v2;
    }
    void insert_line(ll m,ll b){
        auto y=insert({m,b});
        y->succ=[=]{return next(y)==end()?0:&*next(y);};
        if(bad(y)){erase(y);return;}
        while(next(y)!=end()&&bad(next(y)))erase(next(y));
        while(y!=begin()&&bad(prev(y)))erase(prev(y));
    }
    ll eval(ll x){
        auto l=*lower_bound((line){x,is_query});
        return l.m*x+l.b;
    }
};
dynamic_hull cht;
memset(dp,0,sizeof(dp));
cht.insert_line(0,0);
dp[i]=p[i]*b[i]-cht.eval(b[i]);
cht.insert_line(p[i-1],-dp[i]);
\end{lstlisting}
\subsection{LIS}
\begin{lstlisting}
vector<int>LIS(vector<int>&a){
    int n=a.size();
    vector<int>lis,par(n,-1),pos(n);
    for(int i=0;i<n;i++){
        auto it=lower_bound(lis.begin(),lis.end(),a[i]);
        int idx=it-lis.begin();
        if(it==lis.end())lis.push_back(a[i]);
        else *it=a[i];
        pos[idx]=i;
        if(idx>0)par[i]=pos[idx-1];
    }
    vector<int>res;
    for(int cur=pos[lis.size()-1];cur!=-1;cur=par[cur])
        res.push_back(a[cur]);
    reverse(res.begin(),res.end());
    return res;
}
\end{lstlisting}
\section{Number Theory}

\subsection{Big MOD}
\begin{lstlisting}
const int MOD=int(1e9)+7;
int64_t fow(int64_t a,int64_t b){
   int64_t ans=1;
   while(b){
      if(b&1){ ans=(ans*a)%MOD;  }
      a=(a*a)%MOD;
      (b>>=1);
   }
   return ans;
}
\end{lstlisting}

\subsection{Sieve}
\begin{lstlisting}
vector<int> pr={2};
void sieve(){
   const int SZ=int(1e5)+7;
   vector<bool> isp(SZ+1,true);
   for(int i=3;i*i<=SZ;i+=2)if(isp[i])for(int j=i*i;j<=SZ;isp[j]=false,j+=(i<<1));
   for(int i=3;i<SZ;i+=2)if(isp[i]) pr.push_back(i);
}
\end{lstlisting}

\subsection{Segmented Sieve}
\begin{lstlisting}
// Generate all primes from l to r using segmented sieve in O((r - l) log (r) + sqrt(r))  // needs sieve(limit)
vector<int64_t> segmented_sieve(int64_t l, int64_t r) {
   if (l == 1)    l++;
   int limit = sqrtl(r);
   while (limit * limit <= r) limit++;
   while (limit * limit > r) limit--;
   auto primes = sieve(limit);
   vector<bool> is_prime(r - l + 1, true);
   for (int64_t p : primes) {
      int64_t start = max(p * p, (l + p - 1) / p * p);
      for (int64_t j = start; j <= r; j += p) {
         is_prime[j - l] = false;
      }
   }
   vector<int64_t> vec;
   for (int64_t i = l; i <= r; ++i) {
      if (is_prime[i - l])    vec.push_back(i);
   }
   return vec;
}
\end{lstlisting}

\subsection{Linear Sieve}
\begin{lstlisting}
const int N =1e5 + 9;
int spf[N];
vector<int> primes;
void sieve() {
  for(int i =2; i < N; i++) {
    if (spf[i] ==0) spf[i] =i, primes.push_back(i);
    int sz =primes.size();
    for (int j =0; j < sz && i * primes[j] < N && primes[j] <=spf[i]; j++) {
      spf[i * primes[j]] =primes[j];
    }
  }
}
\end{lstlisting}

\subsection{Phi and Mobius Function}
\begin{lstlisting}
int phi(int n){
   int result = n;
   for (int i = 2; i * i <= n; i++){
      if (n % i == 0){
         while (n % i == 0)n /= i;
         result -= result / i;
      }
   }
   if (n > 1)result -= result / n;
   return result;
}
// nloglogn
void phi_1_to_n(int n){
   vector<int> phi(n + 1);
   iota(phi.begin(),phi.end(),int(0));
   for (int i = 2; i <= n; i++){
      if (phi[i] == i){
         for (int j = i; j <= n; j += i)
            phi[j] -= phi[j] / i;
      }
   }
}
int mob[N];
void mobius(){
   mob[1] = 1;
   for (int i = 2; i < N; i++){
      mob[i]--;
      for (int j = i + i; j < N; j += i){
         mob[j] -= mob[i];
      }
   }
}
// returns mu[x] O(logn)
ll mu(ll x){
   ll cnt = 0;
   while (x > 1){
      ll cur = 0, d = spf[x];
      while (x % d == 0){
         x /= d; cur++;
         if (cur > 1) return 0;
      }
      cnt++
   }
   if (cnt & 1)return -1;
   else  return 1;
}
\end{lstlisting}

\section{Math}
\subsection{Josephus}
\begin{lstlisting}
// n = total person// will kill every kth person, if k = 2, 2,4,6,...// returns the mth killed person
// O(k log n)
int64_t josephus(int64_t n, int64_t k, int64_t m){
   m = n - m;
   if (k <= 1) return n - m;
   int64_t i = m;
   while (i < n){
      int64_t r = (i - m + k - 2) / (k - 1);
      if ((i + r) > n)r = n - i;
      else if (!r) r = 1;
      i += r;
      m = (m + (r * k)) % i;
   }
   return m + 1;
}
\end{lstlisting}
\subsection{FFT}
\begin{lstlisting}
const int N =3e5 + 9;
const double PI =acos(-1);
struct base {
  double a, b;
  base(double a =0, double b =0) : a(a), b(b) {}
  const base operator + (const base &c) const
    { return base(a + c.a, b + c.b); }
  const base operator - (const base &c) const
    { return base(a - c.a, b - c.b); }
  const base operator * (const base &c) const
    { return base(a * c.a - b * c.b, a * c.b + b * c.a); }
};
void fft(vector<base> &p, bool inv =0) {
  int n =p.size(), i =0;
  for(int j =1; j < n - 1; ++j) {
    for(int k =n >> 1; k > (i ^=k); k >>=1);
    if(j < i) swap(p[i], p[j]);
  }
  for(int l =1, m; (m =l << 1) <=n; l <<=1) {
    double ang =2 * PI / m;
    base wn =base(cos(ang), (inv ? 1. : -1.) * sin(ang)), w;
    for(int i =0, j, k; i < n; i +=m) {
      for(w =base(1, 0), j =i, k =i + l; j < k; ++j, w =w * wn) {
        base t =w * p[j + l];
        p[j + l] =p[j] - t;
        p[j] =p[j] + t;
      }
    }
  }
  if(inv) for(int i =0; i < n; ++i) p[i].a /=n, p[i].b /=n;
}
vector<long long> multiply(vector<int> &a, vector<int> &b) {
  int n =a.size(), m =b.size(), t =n + m - 1, sz =1;
  while(sz < t) sz <<=1;
  vector<base> x(sz), y(sz), z(sz);
  for(int i =0 ; i < sz; ++i) {
    x[i] =i < (int)a.size() ? base(a[i], 0) : base(0, 0);
    y[i] =i < (int)b.size() ? base(b[i], 0) : base(0, 0);
  }
  fft(x), fft(y);
  for(int i =0; i < sz; ++i) z[i] =x[i] * y[i];
  fft(z, 1);
  vector<long long> ret(sz);
  for(int i =0; i < sz; ++i) ret[i] =(long long) round(z[i].a);
  while((int)ret.size() > 1 && ret.back() ==0) ret.pop_back();
  return ret;
}
\end{lstlisting}

\subsection{NTT}
\begin{lstlisting}
const int64_t MOD = 998244353;
const int64_t G = 3;
struct NTT {
   int max_base; vector<int> rev;
   vector<int64_t> root;
   NTT() {
      int tmp = MOD - 1;
      max_base = 0;
      while ((tmp & 1) == 0) tmp >>= 1, max_base++;
      root.resize(1 << max_base);
      root[0] = 1;
      for (int len = 1; len < (1 << max_base); len <<= 1) {
         int64_t w = modpow(G, (MOD - 1) / (len * 2));
         for (int i = 0; i < len; i++) {
            root[len + i] = (i == 0 ? 1 : root[len + i - 1] * w % MOD);
         }
      }
   }
   void ntt(vector<int64_t> &a, bool invert) {
      int n = (int)a.size();
      if ((int)rev.size() != n) {
         rev.resize(n);
         int k = __builtin_ctz(n) - 1;
         for (int i = 0; i < n; i++)
            rev[i] = (rev[i >> 1] >> 1) | ((i & 1) << k);
      }
      for (int i = 0; i < n; i++)
         if (i < rev[i]) swap(a[i], a[rev[i]]);
      for (int len = 1; len < n; len <<= 1) {
         for (int i = 0; i < n; i += len * 2) {
               for (int j = 0; j < len; j++) {
                  int64_t u = a[i + j];
                  int64_t v = a[i + j + len] * root[len + j] % MOD;
                  a[i + j] = (u + v < MOD ? u + v : u + v - MOD);
                  a[i + j + len] = (u - v >= 0 ? u - v : u - v + MOD);
               }
         }
      }
      if (invert) {
         reverse(a.begin() + 1, a.end());
         int64_t inv_n = modpow(n, MOD - 2);
         for (int64_t &x : a) x = (x * inv_n) % MOD;
      }
   }
   vector<int64_t> multiply(const vector<int64_t> &a, const vector<int64_t> &b) {
      if (a.empty() || b.empty()) return {};
      int need = (int)a.size() + (int)b.size() - 1;
      int n = 1;
      while (n < need) n <<= 1;
      vector<int64_t> fa(a.begin(), a.end()), fb(b.begin(), b.end());
      fa.resize(n); fb.resize(n);
      ntt(fa, false);
      ntt(fb, false);
      for (int i = 0; i < n; i++) fa[i] = fa[i] * fb[i] % MOD;
      ntt(fa, true);
      fa.resize(need);
      return fa;
   }
};
static NTT ntt;
vector<int> pow(vector<int>& a, int p) {
   vector<int> res={1};
   while(p) {
      if(p & 1) res = ntt.multiply(res, a);
      a = ntt.multiply(a, a);
      p >>= 1;
   }
   return res;
}
\end{lstlisting}

\subsection{Derangement}
\begin{lstlisting}
int d[N];   d[0] = 1; d[1] = 0;
for (int i = 1; i < N; i++) {
    d[i] = 1LL * (i - 1) * (d[i - 1] + d[i - 2]) % mod;
}
\end{lstlisting}

\subsection{Pythagorean Triplets}
\begin{lstlisting}
const int N=int(1e6);
vector<pair<int,int>> ans(N+1); //ans[c] = {{a, b}},        s.t. a^2 + b^2 = c^2
void Pythagorean_Triplets(){
   int cnt =0;
  for (int m =1; m * m <=N; m++) {
    for (int n =1; n <m; n++) {
      if ((n & 1) ==(m & 1)) continue;
      int a =m * m - n * n, b =2 * m * n, c =m * m + n * n;
      if (__gcd(a, b) > 1) continue;
      for (int k =1; k * c < N; k++) { // use only k =1 to generate only primitive triples 
        ans[c * k].emplace_back(a * k, b * k);
        cnt++;
      }
      // cnt+=N/c;
    }
  }
}
\end{lstlisting}

\subsection{Mat Expo}
\begin{lstlisting}
struct Mat{
   int n,m;    vector<vector<int>> a;  Mat(){}
   Mat(int _n, int _m){
      n=_n;          m=_m; a.assign(n,vector<int> (m,0));
   }
   Mat(vector<vector<int>> v){
      n=v.size();    m=n?v[0].size():0;   a=v;
   }
   inline void make_unit(){
      assert(n==m);
      for(int i=0;i<n;i++){
         for(int j=0;j<n;j++){
            a[i][j]=(i==j);
         }
      }
   }
   inline Mat operator + (const Mat &b){
      assert(n==b.n && m==b.m);
      Mat ans=Mat(n,m);
      for(int i=0;i<n;i++){
         for(int j=0;j<m;j++){
            ans.a[i][j]=(a[i][j]+b.a[i][j])%mod;
         }
      }
      return ans;
   }
   inline Mat operator - (const Mat& b){
      assert(n==b.n && m==b.m);
      Mat ans=Mat(n,m);
      for(int i=0;i<n;i++){
         for(int j=0;j<m;j++){
            ans.a[i][j]=(a[i][j]-b.a[i][j]+mod)%mod;
         }
      }
      return ans;
   }
   inline Mat operator * (const Mat& b){
      assert(m==b.n);
      Mat ans=Mat(n,b.m);
      for(int i=0;i<n;i++){
         for(int j=0;j<b.m;j++){
            for(int k=0;k<m;k++){
               ans.a[i][j]=(ans.a[i][j]+a[i][k]*b.a[k][j]%mod)%mod;
            }
         }
      }
      return ans;
   }
   inline Mat fow(int64_t k){
      assert(n==m);
      Mat ans(n,m),base=*this;
      ans.make_unit();
      while(k){
         if(k&1){
            ans=ans*base;
         }
         base=base*base;
         k>>=1LL;
      }
      return ans;
   }
   inline Mat& operator +=(const Mat& b){ return (*this)=(*this)+b; }
   inline Mat& operator -=(const Mat& b){ return (*this)=(*this)-b; }
   inline Mat& operator *=(const Mat& b){ return (*this)=(*this)*b; }
   inline bool operator ==(const Mat& b){ return a==b.a; }
   inline bool operator !=(const Mat& b){ return a!=b.a; }
};
\end{lstlisting}

\subsection{XOR Basis}
\begin{lstlisting}
const int LOG_K=60;
struct XorBasis{
 vector<int64_t> basis;
 int64_t N=0,tmp=0;
 void add(int64_t x){
  N++; tmp|=x;
  for(auto&i:basis) x=min(x,x^i);
  if(!x) return;
  for(auto&i:basis) if((i^x)<i) i^=x;
  basis.push_back(x);
  sort(basis.begin(),basis.end());
 }
 int64_t size(){ return basis.size(); }
 void clear(){ N=0; tmp=0; basis.clear(); }
 bool possible(int64_t x){
  for(auto&i:basis) x=min(x,x^i);
  return !x;
 }
 int64_t maxxor(int64_t x=0){
  for(auto&i:basis) x=max(x,x^i);
  return x;
 }
 int64_t minxor(int64_t x=0){
  for(auto&i:basis) x=min(x,x^i);
  return x;
 }
 int64_t cntxor(int64_t x){
  if(!possible(x)) return 0;
  return 1LL<<(N-size());
 }
 int64_t sumOfAll(){
  return tmp*(1LL<<(N-1));
 }
 int64_t kth(int64_t k){
  int64_t sz=size();
  if(k>(1LL<<sz)) return -1;
  k--; int64_t ans=0;
  for(int64_t i=0;i<sz;i++) if(k>>i&1) ans^=basis[i];
  return ans;
 }
};
\end{lstlisting}

\subsection{Nth Fibonacci}
\begin{lstlisting}
map<int64_t, int64_t> dp;     // works for n<=1e18
int64_t f(int64_t n) {
    if (n <=2) return n >=1; // depends on problem
    if (dp.count(n)) return dp[n];
    auto a =f(n / 2);
    auto b =f(n / 2 + 1);
    if (n % 2 ==0) {
        dp[n] =((((b*2LL)%mod - (a%mod)+mod))%mod * a)%mod;
    } else {
        dp[n] =((a * a)%mod + (b * b)%mod)%mod;
    }
    return dp[n];
}
\end{lstlisting}

\subsection{Discrete and Primitive Root}
\begin{lstlisting}
// Primitive root modulo p (p must be prime) g is primitive if g^(phi/p_i) != 1 for all prime factors p_i of phi 
int PrimitiveRoot(int p){
    vector<int>fact;
    int phi=p-1,n=phi;
    for(int i=2;i*i<=n;i++){
        if(n%i==0){
            fact.push_back(i);
            while(n%i==0)n/=i;
        }
    }
    if(n>1)fact.push_back(n);
    for(int g=2;g<=p;g++){
        bool ok=true;
        for(int f:fact)
            ok&=BigMod(g,phi/f,p)!=1;
        if(ok)return g;
    }
    return -1;
}

// Solve x to power k ≡ a (mod n), n prime Prints all solutions using primitive root + BSGS 
void printDiscreteRoot(int k,int a,int n){
    if(a==0){cout<<"1\n0\n";return;}
    int g=PrimitiveRoot(n),phi=n-1;
    int sq=(int)sqrt(n)+1;
    vector<pair<int,int>>dec(sq);
    for(int i=1;i<=sq;i++)
        dec[i-1]={BigMod(g,1LL*i*sq%phi*k%phi,n),i};
    sort(dec.begin(),dec.end());
    int any=-1;
    for(int i=0;i<sq;i++){
        int val=1LL*BigMod(g,1LL*i*k%phi,n)*a%n;
        auto it=lower_bound(dec.begin(),dec.end(),make_pair(val,0));
        if(it!=dec.end()&&it->first==val){
            any=it->second*sq-i;
            break;
        }
    }
    if(any==-1){cout<<"0\n";return;}
    int delta=phi/__gcd(k,phi);
    vector<int>ans;
    for(int cur=(any%delta+delta)%delta;cur<phi;cur+=delta)
        ans.push_back(powmod(g,cur,n));
    sort(ans.begin(),ans.end());
    cout<<ans.size()<<"\n";
    for(int x:ans)cout<<x<<" ";
}
\end{lstlisting}

\subsection{Linear diphantine Eqn}
\begin{lstlisting}
// Extended Euclidean Algorithm to find gcd and coefficients x, y
ll extended_euclid(ll a,ll b,ll &x,ll &y){
    ll xx=y=0, yy=x=1;
    while(b){
        ll q=a/b,t=b; b=a%b; a=t;
        t=xx; xx=x-q*xx; x=t;
        t=yy; yy=y-q*yy; y=t;
    }
    return a;
}
// Solves a*x + b*y = c, returns true if possible, x0 and y0 are any solution
bool find_any_solution(ll a,ll b,ll c,ll &x0,ll &y0,ll &g){
    if(a==0 && b==0){ 
        if(c) return false;
        x0=y0=g=0; return true;
    }
    g=extended_euclid(abs(a),abs(b),x0,y0);
    if(c%g!=0) return false;
    x0*=c/g; y0*=c/g;
    if(a<0) x0*=-1;
    if(b<0) y0*=-1;
    return true;
}
// Shifts solution by cnt steps: x += cnt*b, y -= cnt*a
void shift_solution(ll &x,ll &y,ll a,ll b,ll cnt){
    x+=cnt*b; y-=cnt*a;
}
// Finds number of integer solutions within bounds for x and y
ll find_all_solutions(ll a,ll b,ll c,ll minx,ll maxx,ll miny,ll maxy){
    ll x,y,g;
    if(!find_any_solution(a,b,c,x,y,g)) return 0;
    if(a==0 && b==0){ assert(c==0); return 1LL*(maxx-minx+1)*(maxy-miny+1);}
    if(a==0) return (maxx-minx+1)*(miny<=c/b && c/b<=maxy);
    if(b==0) return (maxy-miny+1)*(minx<=c/a && c/a<=maxx);
    a/=g; b/=g;
    ll sign_a=a>0?+1:-1, sign_b=b>0?+1:-1;
    shift_solution(x,y,a,b,(minx-x)/b);
    if(x<minx) shift_solution(x,y,a,b,sign_b);
    if(x>maxx) return 0;
    ll lx1=x;
    shift_solution(x,y,a,b,(maxx-x)/b);
    if(x>maxx) shift_solution(x,y,a,b,-sign_b);
    ll rx1=x;
    shift_solution(x,y,a,b,-(miny-y)/a);
    if(y<miny) shift_solution(x,y,a,b,-sign_a);
    if(y>maxy) return 0;
    ll lx2=x;
    shift_solution(x,y,a,b,-(maxy-y)/a);
    if(y>maxy) shift_solution(x,y,a,b,sign_a);
    ll rx2=x;
    if(lx2>rx2) swap(lx2,rx2);
    ll lx=max(lx1,lx2), rx=min(rx1,rx2);
    if(lx>rx) return 0;
    return (rx-lx)/abs(b)+1;
}
\end{lstlisting}
\subsection{CRT}
\begin{lstlisting}
using T = __int128; // extend_euclid ache
// finds x such that x % m1 = a1, x % m2 = a2. m1 and m2 may not be coprime
// here, x is unique modulo m = lcm(m1, m2). returns (x, m). on failure, m = -1.
pair<T, T> CRT(T a1, T m1, T a2, T m2) {
  T p, q;
  T g = extended_euclid(m1, m2, p, q);
  if (a1 % g != a2 % g) return make_pair(0, -1);
  T m = m1 / g * m2;
  p = (p % m + m) % m;
  q = (q % m + m) % m;
  return make_pair((p * a2 % m * (m1 / g) % m + q * a1 % m * (m2 / g) % m) %  m, m);
}
\end{lstlisting}

\section{Geometry}
\subsection{2D}
\begin{lstlisting}
const int N = 3e5 + 9;
const double inf = 1e100;
const double PI = acos((double)-1.0);
const double eps = 1e-9;
int sign(double x) { return (x > eps) - (x < -eps); }
struct PT {
    double x, y;
    PT() { x = 0, y = 0; }
    PT(double x, double y) : x(x), y(y) {}
    PT(const PT &p) : x(p.x), y(p.y)    {}
    PT operator + (const PT &a) const { return PT(x + a.x, y + a.y); }
    PT operator - (const PT &a) const { return PT(x - a.x, y - a.y); }
    PT operator * (const double a) const { return PT(x * a, y * a); }
    friend PT operator * (const double &a, const PT &b) { return PT(a * b.x, a * b.y); }
    PT operator / (const double a) const { return PT(x / a, y / a); }
    bool operator == (PT a) const { return sign(a.x - x) == 0 && sign(a.y - y) == 0; }
    bool operator != (PT a) const { return !(*this == a); }
    bool operator < (PT a) const { return sign(a.x - x) == 0 ? y < a.y : x < a.x; }
    bool operator > (PT a) const { return sign(a.x - x) == 0 ? y > a.y : x > a.x; }
    double norm() { return sqrt(x * x + y * y); }
    double norm2() { return x * x + y * y; }
    PT perp() { return PT(-y, x); }
    double arg() { return atan2(y, x); }
    PT truncate(double r) { // returns a vector with norm r and having same direction
        double k = norm();
        if (!sign(k)) return *this;
        r /= k;
        return PT(x * r, y * r);
    }
};
istream &operator >> (istream &in, PT &p) { return in >> p.x >> p.y; }
ostream &operator << (ostream &out, PT &p) { return out << "(" << p.x << "," << p.y << ")"; }
inline double dot(PT a, PT b) { return a.x * b.x + a.y * b.y; }
inline double dist2(PT a, PT b) { return dot(a - b, a - b); }
inline double dist(PT a, PT b) { return sqrt(dot(a - b, a - b)); }
inline double cross(PT a, PT b) { return a.x * b.y - a.y * b.x; }
inline double cross2(PT a, PT b, PT c) { return cross(b - a, c - a); }
inline int orientation(PT a, PT b, PT c) { return sign(cross(b - a, c - a)); }
PT perp(PT a) { return PT(-a.y, a.x); }
PT rotateccw90(PT a) { return PT(-a.y, a.x); }
PT rotatecw90(PT a) { return PT(a.y, -a.x); }
PT rotateccw(PT a, double t) { return PT(a.x * cos(t) - a.y * sin(t), a.x * sin(t) + a.y * cos(t)); }
PT rotatecw(PT a, double t) { return PT(a.x * cos(t) + a.y * sin(t), -a.x * sin(t) + a.y * cos(t)); }
double SQ(double x) { return x * x; }
double rad_to_deg(double r) { return (r * 180.0 / PI); }
double deg_to_rad(double d) { return (d * PI / 180.0); }
// returns in rad
double get_angle(PT a, PT b) {
    double costheta = dot(a, b) / a.norm() / b.norm();
    return acos(max((double)-1.0, min((double)1.0, costheta)));
}
struct line {
    PT a, b; // goes through points a and b
    PT v; double c;  //line form: direction vec [cross] (x, y) = c 
    line() {}
    //direction vector v and offset c
	line(PT v, double c) : v(v), c(c) {
        auto p = get_points();
        a = p.first; b = p.second;
	}
	// equation ax + by + c = 0
	line(double _a, double _b, double _c) : v({_b, -_a}), c(-_c) {
		auto p = get_points();
        a = p.first; b = p.second;
	}
	// goes through points p and q
	line(PT p, PT q) : v(q - p), c(cross(v, p)), a(p), b(q) {}
    	pair<PT, PT> get_points() { //extract any two points from this line
		PT p, q; double a = -v.y, b = v.x; // ax + by = c
		if (sign(a) == 0) {
		    p = PT(0, c / b);
		    q = PT(1, c / b);
		}
		else if (sign(b) == 0) {
		    p = PT(c / a, 0);
		    q = PT(c / a, 1);
		}
		else {
		    p = PT(0, c / b);
		    q = PT(1, (c - a) / b);
		}
		return {p, q};
    	}
    // ax + by + c = 0
    array<double, 3> get_abc() {
        double a = -v.y, b = v.x;
        return {a, b, -c};
    }
    // 1 if on the left, -1 if on the right, 0 if on the line
    int side(PT p) { return sign(cross(v, p) - c); }
    // line that is perpendicular to this and goes through point p
    line perpendicular_through(PT p) { return {p, p + perp(v)}; }
    // translate the line by vector t i.e. shifting it by vector t
    line translate(PT t) { return {v, c + cross(v, t)}; }
    // compare two points by their orthogonal projection on this line
    // a projection point comes before another if it comes first according to vector v
    bool cmp_by_projection(PT p, PT q) { return dot(v, p) < dot(v, q); }
	line shift_left(double d) {
		PT z = v.perp().truncate(d);
		return line(a + z, b + z);
	}
};
// minimum distance from point c to line through a and b
double dist_from_point_to_line(PT a, PT b, PT c) {
    return fabs(cross(b - a, c - a) / (b - a).norm());
}
// returns true if  point p is on line segment ab
bool is_point_on_seg(PT a, PT b, PT p) {
    if (fabs(cross(p - b, a - b)) < eps) {
        if (p.x < min(a.x, b.x) - eps || p.x > max(a.x, b.x) + eps) return false;
        if (p.y < min(a.y, b.y) - eps || p.y > max(a.y, b.y) + eps) return false;
        return true;
    }
    return false;
}
// minimum distance point from point c to segment ab that lies on segment ab
PT project_from_point_to_seg(PT a, PT b, PT c) {
    double r = dist2(a, b);
    if (sign(r) == 0) return a;
    r = dot(c - a, b - a) / r;
    if (r < 0) return a;
    if (r > 1) return b;
    return a + (b - a) * r;
}
// minimum distance from point c to segment ab
double dist_from_point_to_seg(PT a, PT b, PT c) {
    return dist(c, project_from_point_to_seg(a, b, c));
}
// 0 if not parallel, 1 if parallel, 2 if collinear
int is_parallel(PT a, PT b, PT c, PT d) {
    double k = fabs(cross(b - a, d - c));
    if (k < eps){
        if (fabs(cross(a - b, a - c)) < eps && fabs(cross(c - d, c - a)) < eps) return 2;
        else return 1;
    }
    else return 0;
}
// 1 if point is ccw to the line, 2 if point is cw to the line, 3 if point is on the line
int point_line_relation(PT a, PT b, PT p) {
    int c = sign(cross(p - a, b - a));
    if (c < 0) return 1;
    if (c > 0) return 2;
    return 3;
}
// intersection point between ab and cd assuming unique intersection exists
bool line_line_intersection(PT a, PT b, PT c, PT d, PT &ans) {
    double a1 = a.y - b.y, b1 = b.x - a.x, c1 = cross(a, b);
    double a2 = c.y - d.y, b2 = d.x - c.x, c2 = cross(c, d);
    double det = a1 * b2 - a2 * b1;
    if (det == 0) return 0;
    ans = PT((b1 * c2 - b2 * c1) / det, (c1 * a2 - a1 * c2) / det);
    return 1;
}
// intersection point between segment ab and segment cd assuming unique intersection exists
bool seg_seg_intersection(PT a, PT b, PT c, PT d, PT &ans) {
    double oa = cross2(c, d, a), ob = cross2(c, d, b);
    double oc = cross2(a, b, c), od = cross2(a, b, d);
    if (oa * ob < 0 && oc * od < 0){
        ans = (a * ob - b * oa) / (ob - oa);
        return 1;
    }
    else return 0;
}
// mychanges
// even if no unique intersection, meaning in the same direction
bool seg_seg_intersection(PT a, PT b, PT c, PT d) {
    double oa = cross2(c, d, a), ob = cross2(c, d, b);
    double oc = cross2(a, b, c), od = cross2(a, b, d);
    if (oa * ob < 0 && oc * od < 0){
        return true;
    }
    // checking orientation for very precise calculation
    else if(oa == 0 && is_point_on_seg(c, d, a))return true;
    else if(ob == 0 && is_point_on_seg(c, d, b))return true;
    else if(oc == 0 && is_point_on_seg(a, b, c))return true;
    else if(od == 0 && is_point_on_seg(a, b, d))return true;
    else return false;
}
// intersection  between segment ab and line cd
// 0 if do not intersect, 1 if proper intersect, 2 if segment intersect
int seg_line_relation(PT a, PT b, PT c, PT d) {
    double p = cross2(c, d, a);
    double q = cross2(c, d, b);
    if (sign(p) == 0 && sign(q) == 0) return 2;
    else if (p * q < 0) return 1;
    else return 0;
}
// intersection between segament ab and line cd assuming unique intersection exists
bool seg_line_intersection(PT a, PT b, PT c, PT d, PT &ans) {
    bool k = seg_line_relation(a, b, c, d);
    assert(k != 2);
    if (k) line_line_intersection(a, b, c, d, ans);
    return k;
}
// minimum distance from segment ab to segment cd
double dist_from_seg_to_seg(PT a, PT b, PT c, PT d) {
    PT dummy;
    if (seg_seg_intersection(a, b, c, d, dummy)) return 0.0;
    else return min({dist_from_point_to_seg(a, b, c), dist_from_point_to_seg(a, b, d), 
        dist_from_point_to_seg(c, d, a), dist_from_point_to_seg(c, d, b)});
}
struct circle {
    PT p; double r;
    circle() {}
    circle(PT _p, double _r): p(_p), r(_r) {};
    // center (x, y) and radius r
    circle(double x, double y, double _r): p(PT(x, y)), r(_r) {};
    // circumcircle of a triangle
    // the three points must be unique
    circle(PT a, PT b, PT c) {
        b = (a + b) * 0.5;
        c = (a + c) * 0.5;
        line_line_intersection(b, b + rotatecw90(a - b), c, c + rotatecw90(a - c), p);
        r = dist(a, p);
    }
    // inscribed circle of a triangle
    // pass a bool just to differentiate from circumcircle
    circle(PT a, PT b, PT c, bool t) {
        line u, v;
        double m = atan2(b.y - a.y, b.x - a.x), n = atan2(c.y - a.y, c.x - a.x);
        u.a = a;
        u.b = u.a + (PT(cos((n + m)/2.0), sin((n + m)/2.0)));
        v.a = b;
        m = atan2(a.y - b.y, a.x - b.x), n = atan2(c.y - b.y, c.x - b.x);
        v.b = v.a + (PT(cos((n + m)/2.0), sin((n + m)/2.0)));
        line_line_intersection(u.a, u.b, v.a, v.b, p);
        r = dist_from_point_to_seg(a, b, p);
    }
    bool operator == (circle v) { return p == v.p && sign(r - v.r) == 0; }
    double area() { return PI * r * r; }
    double circumference() { return 2.0 * PI * r; }
};
//0 if outside, 1 if on circumference, 2 if inside circle
int circle_point_relation(PT p, double r, PT b) {
    double d = dist(p, b);
    if (sign(d - r) < 0) return 2;
    if (sign(d - r) == 0) return 1;
    return 0;
}
// 0 if outside, 1 if on circumference, 2 if inside circle
int circle_line_relation(PT p, double r, PT a, PT b) {
    double d = dist_from_point_to_line(a, b, p);
    if (sign(d - r) < 0) return 2;
    if (sign(d - r) == 0) return 1;
    return 0;
}
//compute intersection of line through points a and b with
//circle centered at c with radius r > 0
vector<PT> circle_line_intersection(PT c, double r, PT a, PT b) {
    vector<PT> ret;
    b = b - a; a = a - c;
    double A = dot(b, b), B = dot(a, b);
    double C = dot(a, a) - r * r, D = B * B - A * C;
    if (D < -eps) return ret;
    ret.push_back(c + a + b * (-B + sqrt(D + eps)) / A);
    if (D > eps) ret.push_back(c + a + b * (-B - sqrt(D)) / A);
    return ret;
}
//5 - outside and do not intersect
//4 - intersect outside in one point
//3 - intersect in 2 points
//2 - intersect inside in one point
//1 - inside and do not intersect
int circle_circle_relation(PT a, double r, PT b, double R) {
    double d = dist(a, b);
    if (sign(d - r - R) > 0)  return 5;
    if (sign(d - r - R) == 0) return 4;
    double l = fabs(r - R);
    if (sign(d - r - R) < 0 && sign(d - l) > 0) return 3;
    if (sign(d - l) == 0) return 2;
    if (sign(d - l) < 0) return 1;
    assert(0); return -1;
}
vector<PT> circle_circle_intersection(PT a, double r, PT b, double R) {
    if (a == b && sign(r - R) == 0) return {PT(1e18, 1e18)};
    vector<PT> ret;
    double d = sqrt(dist2(a,  b));
    if (d > r + R || d + min(r,  R) < max(r,  R)) return ret;
    double x = (d * d - R * R + r * r) / (2 * d);
    double y = sqrt(r * r - x * x);
    PT v = (b - a) / d;
    ret.push_back(a + v * x  +  rotateccw90(v) * y);
    if (y > 0) ret.push_back(a + v * x - rotateccw90(v) * y);
    return ret;
}
// returns two circle c1, c2 through points a, b and of radius r
// 0 if there is no such circle, 1 if one circle, 2 if two circle
int get_circle(PT a, PT b, double r, circle &c1, circle &c2) {
    vector<PT> v = circle_circle_intersection(a, r, b, r);
    int t = v.size();
    if (!t) return 0;
    c1.p = v[0], c1.r = r;
    if (t == 2) c2.p = v[1], c2.r = r;
    return t;
}
// returns area of intersection between two circles
double circle_circle_area(PT a, double r1, PT b, double r2) {
    double d = (a - b).norm();
    if(r1 + r2 < d + eps) return 0;
    if(r1 + d < r2 + eps) return PI * r1 * r1;
    if(r2 + d < r1 + eps) return PI * r2 * r2;
    double theta_1 = acos((r1 * r1 + d * d - r2 * r2) / (2 * r1 * d)), 
    	theta_2 = acos((r2 * r2 + d * d - r1 * r1)/(2 * r2 * d));
    return r1 * r1 * (theta_1 - sin(2 * theta_1)/2.) + r2 * r2 * (theta_2 - sin(2 * theta_2)/2.);
}
double area_of_triangle(PT a, PT b, PT c) {
    return fabs(cross(b - a, c - a) * 0.5);
}
// -1 if strictly inside, 0 if on the polygon, 1 if strictly outside
int is_point_in_triangle(PT a, PT b, PT c, PT p) {
    if (sign(cross(b - a,c - a)) < 0) swap(b, c);
    int c1 = sign(cross(b - a,p - a));
    int c2 = sign(cross(c - b,p - b));
    int c3 = sign(cross(a - c,p - c));
    if (c1<0 || c2<0 || c3 < 0) return 1;
    if (c1 + c2 + c3 != 3) return 0;
    return -1;
}
double area(vector<PT> &p) {
    double ans = 0; int n = p.size();
    for (int i = 0; i < n; i++) ans += cross(p[i], p[(i + 1) % n]);
    return fabs(ans) * 0.5;
}
// 0 if cw, 1 if ccw
bool get_direction(vector<PT> &p) {
    double ans = 0; int n = p.size();
    for (int i = 0; i < n; i++) ans += cross(p[i], p[(i + 1) % n]);
    if (sign(ans) > 0) return 1;
    return 0;
}
// returns the convex hull of the set of points p
vector<PT> convex_hull(vector<PT> &p) {
	if (p.size() <= 1) return p;
	vector<PT> v = p;
    sort(v.begin(), v.end());
    vector<PT> up, dn;
    for (auto& p : v) {
        while (up.size() > 1 && orientation(up[up.size() - 2], up.back(), p) >= 0) {
            up.pop_back();
        }
        while (dn.size() > 1 && orientation(dn[dn.size() - 2], dn.back(), p) <= 0) {
            dn.pop_back();
        }
        up.push_back(p);
        dn.push_back(p);
    }
    v = dn;
    if (v.size() > 1) v.pop_back();
    reverse(up.begin(), up.end());
    up.pop_back();
    for (auto& p : up) {
        v.push_back(p);
    }
    if (v.size() == 2 && v[0] == v[1]) v.pop_back();
    return v;
}
 //checks if convex or not
bool is_convex(vector<PT> &p) {
    bool s[3]; s[0] = s[1] = s[2] = 0;
    int n = p.size();
    for (int i = 0; i < n; i++) {
        int j = (i + 1) % n;
        int k = (j + 1) % n;
        s[sign(cross(p[j] - p[i], p[k] - p[i])) + 1] = 1;
        if (s[0] && s[2]) return 0;
    }
    return 1;
}
// -1 if strictly inside, 0 if on the polygon, 1 if strictly outside
// it must be strictly convex, otherwise make it strictly convex first
int is_point_in_convex(vector<PT> &p, const PT& x) { // O(log n)
    int n = p.size(); assert(n >= 3);
    int a = orientation(p[0], p[1], x), b = orientation(p[0], p[n - 1], x);
    if (a < 0 || b > 0) return 1;
    int l = 1, r = n - 1;
    while (l + 1 < r) {
        int mid = l + r >> 1;
        if (orientation(p[0], p[mid], x) >= 0) l = mid;
        else r = mid;
    }
    int k = orientation(p[l], p[r], x);
    if (k <= 0) return -k;
    if (l == 1 && a == 0) return 0;
    if (r == n - 1 && b == 0) return 0;
    return -1;
}
bool is_point_on_polygon(vector<PT> &p, const PT& z) {
    int n = p.size();
    for (int i = 0; i < n; i++) {
    	if (is_point_on_seg(p[i], p[(i + 1) % n], z)) return 1;
    }
    return 0;
}
// returns 1e9 if the point is on the polygon 
int winding_number(vector<PT> &p, const PT& z) { // O(n)
    if (is_point_on_polygon(p, z)) return 1e9;
    int n = p.size(), ans = 0;
    for (int i = 0; i < n; ++i) {
        int j = (i + 1) % n;
        bool below = p[i].y < z.y;
        if (below != (p[j].y < z.y)) {
            auto orient = orientation(z, p[j], p[i]);
            if (orient == 0) return 0;
            if (below == (orient > 0)) ans += below ? 1 : -1;
        }
    }
    return ans;
}
// -1 if strictly inside, 0 if on the polygon, 1 if strictly outside
int is_point_in_polygon(vector<PT> &p, const PT& z) { // O(n)
    int k = winding_number(p, z);
    return k == 1e9 ? 0 : k == 0 ? 1 : -1;
}
// id of the vertex having maximum dot product with z
// polygon must need to be convex
// top - upper right vertex
// for minimum dot product negate z and return -dot(z, p[id])
int extreme_vertex(vector<PT> &p, const PT &z, const int top) { // O(log n)
    int n = p.size();
    if (n == 1) return 0;
	double ans = dot(p[0], z); int id = 0;
    if (dot(p[top], z) > ans) ans = dot(p[top], z), id = top;
    int l = 1, r = top - 1;
    while (l < r) {
        int mid = l + r >> 1;
        if (dot(p[mid + 1], z) >= dot(p[mid], z)) l = mid + 1;
        else r = mid;
    }
    if (dot(p[l], z) > ans) ans = dot(p[l], z), id = l;
    l = top + 1, r = n - 1;
    while (l < r) {
        int mid = l + r >> 1;
        if (dot(p[(mid + 1) % n], z) >= dot(p[mid], z)) l = mid + 1;
        else r = mid;
    }
    l %= n;
    if (dot(p[l], z) > ans) ans = dot(p[l], z), id = l;
    return id;
}
pair<PT, int> point_poly_tangent(vector<PT> &p, PT Q, int dir, int l, int r) {
    while (r - l > 1) {
        int mid = (l + r) >> 1;
        bool pvs = orientation(Q, p[mid], p[mid - 1]) != -dir;
        bool nxt = orientation(Q, p[mid], p[mid + 1]) != -dir;
        if (pvs && nxt) return {p[mid], mid};
        if (!(pvs || nxt)) {
            auto p1 = point_poly_tangent(p, Q, dir, mid + 1, r);
            auto p2 = point_poly_tangent(p, Q, dir, l, mid - 1);
            return orientation(Q, p1.first, p2.first) == dir ? p1 : p2;
        }
        if (!pvs) {
            if (orientation(Q, p[mid], p[l]) == dir)  r = mid - 1;
            else if (orientation(Q, p[l], p[r]) == dir) r = mid - 1;
            else l = mid + 1;
        }
        if (!nxt) {
            if (orientation(Q, p[mid], p[l]) == dir)  l = mid + 1;
            else if (orientation(Q, p[l], p[r]) == dir) r = mid - 1;
            else l = mid + 1;
        }
    }
    pair<PT, int> ret = {p[l], l};
    for (int i = l + 1; i <= r; i++) ret = orientation(Q, ret.first, p[i]) != dir ? make_pair(p[i], i) : ret;
    return ret;
}
// (ccw, cw) tangents from a point that is outside this convex polygon
// returns indexes of the points
// ccw means the tangent from Q to that point is in the same direction as the polygon ccw direction
pair<int, int> tangents_from_point_to_polygon(vector<PT> &p, PT Q){
    int ccw = point_poly_tangent(p, Q, 1, 0, (int)p.size() - 1).second;
    int cw = point_poly_tangent(p, Q, -1, 0, (int)p.size() - 1).second;
    return make_pair(ccw, cw);
}
// minimum distance from a point to a convex polygon
// it assumes point lie strictly outside the polygon
double dist_from_point_to_polygon(vector<PT> &p, PT z) {
    double ans = inf;
    int n = p.size();
    if (n <= 3) {
        for(int i = 0; i < n; i++) ans = min(ans, dist_from_point_to_seg(p[i], p[(i + 1) % n], z));
        return ans;
    }
    auto [r, l] = tangents_from_point_to_polygon(p, z);
    if(l > r) r += n;
    while (l < r) {
        int mid = (l + r) >> 1;
        double left = dist2(p[mid % n], z), right= dist2(p[(mid + 1) % n], z);
        ans = min({ans, left, right});
        if(left < right) r = mid;
        else l = mid + 1;
    }
    ans = sqrt(ans);
    ans = min(ans, dist_from_point_to_seg(p[l % n], p[(l + 1) % n], z));
    ans = min(ans, dist_from_point_to_seg(p[l % n], p[(l - 1 + n) % n], z));
    return ans;
}
// minimum distance from convex polygon p to line ab
// returns 0 is it intersects with the polygon
// top - upper right vertex
double dist_from_polygon_to_line(vector<PT> &p, PT a, PT b, int top) { //O(log n)
	PT orth = (b - a).perp();
	if (orientation(a, b, p[0]) > 0) orth = (a - b).perp();
	int id = extreme_vertex(p, orth, top);
	if (dot(p[id] - a, orth) > 0) return 0.0; //if orth and a are in the same half of the line, then poly and line intersects
	return dist_from_point_to_line(a, b, p[id]); //does not intersect
}
// minimum distance from a convex polygon to another convex polygon
// the polygon doesnot overlap or touch
// tested in https://toph.co/p/the-wall
double dist_from_polygon_to_polygon(vector<PT> &p1, vector<PT> &p2) { // O(n log n)
    double ans = inf;
    for (int i = 0; i < p1.size(); i++) {
        ans = min(ans, dist_from_point_to_polygon(p2, p1[i]));
    }
    for (int i = 0; i < p2.size(); i++) {
        ans = min(ans, dist_from_point_to_polygon(p1, p2[i]));
    }
    return ans;
}
// maximum distance from a convex polygon to another convex polygon
double maximum_dist_from_polygon_to_polygon(vector<PT> &u, vector<PT> &v){ //O(n)
    int n = (int)u.size(), m = (int)v.size();
    double ans = 0;
    if (n < 3 || m < 3) {
        for (int i = 0; i < n; i++) {
            for (int j = 0; j < m; j++) ans = max(ans, dist2(u[i], v[j]));
        }
        return sqrt(ans);
    }
    if (u[0].x > v[0].x) swap(n, m), swap(u, v);
    int i = 0, j = 0, step = n + m + 10;
    while (j + 1 < m && v[j].x < v[j + 1].x) j++ ;
    while (step--) {
        if (cross(u[(i + 1)%n] - u[i], v[(j + 1)%m] - v[j]) >= 0) j = (j + 1) % m;
        else i = (i + 1) % n;
        ans = max(ans, dist2(u[i], v[j]));
    }
    return sqrt(ans);
}
// mychanges
// did not test but should work
double polygon_area(vector<PT>& v){
    double area = 0;
    int n = v.size();
    for(int i = 0; i < n; i++){
        PT a = v[i], b = v[(i + 1) % n];
        int x1 = a.x, y1 = a.y;
        int x2 = b.x, y2 = b.y;
        area += (x1 * y2 - x2 * y1);
    }
    area = abs(area);
    area = area / 2.0; // this can give precision errors if value is so much beacuse double have big value precision errors
    return area;
}
// mychanges
// exist named is_point_in_polygon
int position_of_a_point_in_a_polygon(vector<PT>& v, PT c){
    double xInf = inf, yInf = c.y + 1;
    PT d(xInf, yInf);
    int n = v.size();
    int cnt = 0;
    for(int i = 0; i < n; i++){
        PT a = v[i], b = v[(i + 1) % n];
        int val = seg_seg_intersection(a, b, c, d);
        if(val == 1)cnt++;
        else if(val == 2)return 0; // BOUNDARY
    }
    if(cnt % 2)return 1;    // INSIDE
    else return -1;         // OUTSIDE
}
int getSign(int x1,int y1, int x2, int y2, int x3, int y3){
    // int direct = x1 * y2 + x2 * y3 + x3 * y1 - x1 * y3 - x2 * y1 - x3 * y2;
    int direct = (y2 - y1) * (x3 - x2) - (y3 - y2) * (x2 - x1);
    if(direct > 0)return 1;     // clockWise
    if(direct < 0)return 2;     // counter - ClockWise
    if(direct == 0)return 0;    // cointinear
}
// Function to find orientation of ordered triplet (x1,y1), (x2,y2), (x3,y3)
// Returns: 0 -> collinear, 1 -> clockwise, 2 -> counter-clockwise
int ori(int x1, int y1, int x2, int y2, int x3, int y3) {
    int val = (y2 - y1) * (x3 - x2) - (x2 - x1) * (y3 - y2);
    if (val == 0) return 1;        // clockwise
    else if (val < 0) return 2;    // counter-clockwise
    return 0;                      // collinear
}
\end{lstlisting}


\subsection{Closest Pair}
\begin{lstlisting}
#define x first
#define y second
long long dist2(pair<int, int> a, pair<int, int> b) {
 return 1LL * (a.x - b.x) * (a.x - b.x) + 1LL * (a.y - b.y) * (a.y - b.y);}pair<int, int> closest_pair(vector<pair<int, int>> a) {
 int n = a.size();
 assert(n >= 2);
 vector<pair<pair<int, int>, int>> p(n);
 for (int i = 0; i < n; i++) p[i] = {a[i], i};
 sort(p.begin(), p.end());
 int l = 0, r = 2;
 long long ans = dist2(p[0].x, p[1].x);
 pair<int, int> ret = {p[0].y, p[1].y};
 while (r < n) {
  while (l < r && 1LL * (p[r].x.x - p[l].x.x) * (p[r].x.x - p[l].x.x) >= ans) l++;
  for (int i = l; i < r; i++) {
   long long nw = dist2(p[i].x, p[r].x);
   if (nw < ans) {
    ans = nw;
    ret = {p[i].y, p[r].y};}}r++;}
 return ret;}
\end{lstlisting}

\subsection{Pick's Theorem}
\begin{lstlisting}
theorem : area = In + (on_boundary / 2) - 1;
in = strictly inside polygon
T point(PT a, PT b){
   T x = abs(a.x - b.x);
   T y = abs(a.y - b.y);
   return __gcd(x, y) - 1;
}
main(){
   ll on_boundary = n, area = 0LL;
   for (i = 0; i < n; i++){
      area += cross(p[i], p[(i + 1) % n]);
      on_boundary += point(p[i], p[(i + 1) % n]);
   }
}
\end{lstlisting}
\subsection{Sweep Line}
\begin{lstlisting}
long long sweep_line_x;
struct Segment{long long a,b,x,y;int idx;};
struct Point{long long x,y;int segment_idx;};
int sign(long long x){return x==0?0:(x<0?-1:1);}
double coordinate(const Segment&s){
    if(s.a==s.x)return s.b;
    return s.b+(double)(s.y-s.b)*(sweep_line_x-s.a)/(s.x-s.a);
}
bool operator<(const Segment&u,const Segment&v){
    if(u.idx==v.idx)return false;
    return coordinate(u)<coordinate(v);
}
bool operator==(const Segment&u,const Segment&v){return u.idx==v.idx;}
long long orientation(Point a,Point b,Point c){
    return (b.x-a.x)*(c.y-a.y)-(c.x-a.x)*(b.y-a.y);
}
bool intersect(const Segment&a,const Segment&b){
    Point p1{a.a,a.b,0},q1{a.x,a.y,0},p2{b.a,b.b,0},q2{b.x,b.y,0};
    long long o1=orientation(p1,q1,p2),o2=orientation(p1,q1,q2);
    long long o3=orientation(p2,q2,p1),o4=orientation(p2,q2,q1);
    return sign(o1)*sign(o2)<=0&&sign(o3)*sign(o4)<=0;
}
int main(){
    ios::sync_with_stdio(false);
    cin.tie(nullptr);
    int n;cin>>n;
    vector<Segment>segments(n);
    vector<Point>events;
    for(int i=0;i<n;i++){
        cin>>segments[i].a>>segments[i].b>>segments[i].x>>segments[i].y;
        segments[i].idx=i;
        events.push_back({segments[i].a,segments[i].b,i});
        events.push_back({segments[i].x,segments[i].y,i});
    }
    sort(events.begin(),events.end(),[](const Point&p,const Point&q){
        if(p.x!=q.x)return p.x<q.x;
        return p.y<q.y;
    });
    set<Segment>active;
    int first=-1,second=-1;
    for(auto&e:events){
        sweep_line_x=e.x;
        int id=e.segment_idx;
        auto it=active.find(segments[id]);
        if(it!=active.end()){
            auto nxt=next(it);
            if(it!=active.begin()&&nxt!=active.end()){
                auto prv=prev(it);
                if(intersect(segments[prv->idx],segments[nxt->idx])){
                    first=prv->idx;second=nxt->idx;break;
                }
            }
            active.erase(it);
        }else{
            it=active.lower_bound(segments[id]);
            if(it!=active.end()&&intersect(segments[id],*it)){
                first=id;second=it->idx;break;
            }
            if(it!=active.begin()){
                auto prv=prev(it);
                if(intersect(segments[id],*prv)){
                    first=id;second=prv->idx;break;
                }
            }
            active.insert(segments[id]);
        }
    }
    if(first>second)swap(first,second);
    int cnt=0;
    for(int i=0;i<n;i++)
        if(i!=second&&intersect(segments[i],segments[second]))cnt++;
    cout<<(cnt>1?second+1:first+1);
}
\end{lstlisting}

\section{Miscellaneous}
\subsection{Formulas}
\begin{lstlisting}
* Sum of n natural numbers = n * (n + 1) / 2;
* Sum of n natural squared numbers = n * (n + 1) * (2 * n + 1) / 6;
* Sum of n natural cube numbers = (n * (n + 1) / 2)^2;
* Sum of first n even numbers = n * (n + 1)
* Sum of first n odd numbers = n ^ 2;
* Chord Length = 2 * sqrt(r*r - d*d)
* area of a sector/arc = (T/360) * PI * r * r (T is degree)
* area of a sector/arc = (1/2) * T * r * r (T is radians)
* sine rule: a/sin A = b/sin B = c/sin C
* cosine rule: c * c = a * a + b * b - 2ab cos C
* sum of angles of a polygon (n - 2) * 180
* Catalen numbers Cn = (1 / (n + 1)) * (2n C n) = (2n C n) - (2n   C (n + 1)) = sumOf i to n (Ci * C(n-i))
* Area of isoceles = 0.5 * a * b * sin(c)
* Area of any triangle = sqrt(s(s-a)(s-b)(s-c))
* Total angle of polygon = (n-2)*180
* Picks theorem = Area = I + B/2 - 1
* Arithmetic series nth number, an = a1 + (n - 1) * d
* Arithmetic series sum, S_n = (n/2) * (a_1 + a_n)
* Geometric series nth number, an = a1 * r^(n-1)
* Geometric series sum, S_n = a* (1 - r^n) / (r - 1)
* Area of trapezium without height = 1/2 * (a + b) * h
* Right circular cone volume, V = PI * r^2 * (h / 3)
* conical frustum, V = (PI/3) * height * (r1 * r1 + r2 * r2 + (r1 * r2))
* Stars and bars = (n + k - 1) C (k - 1)
* Derangements, D(n) = (n - 1) * (D(n - 1) + D(n - 2))
// BITS
* Lowest set bit = x & (-x)
* Remove lowest set bit = x & (x - 1)
* (a | b) =(a^b)|(a&b)       * a^(a&b) =(a|b)^b
* b^(a&b) =(a|b)^a           * (a&b)^(a|b) =(a^b)
* (a+b) =(a^b) + 2*(a&b)     * (a+b) =(a|b) + (a&b) //plus
* (a-b) =(a^(a&b))-((a|b)^a) * (a-b) =(a^(a&b))-(b^(a&b)) 
* (a-b) =((a|b)^b)-(b^(a&b)) * (a-b) =((a|b)^b)-((a|b)^a)
// Primes
1e18+21, 1e18+33, 1e18+87, 1e18+93, 2e18+11, 2e18+33, 
2e18+63, 2e18+87
const int Primes[50] = {2487161969, 3856247663, 3158177263, 4443359849, 2928712063, 1534911193, 3704798833, 3858635329, 1406144053, 1799020313, 4161289499, 4808093141, 1074412681, 1622517073, 1910710603, 3563848823, 4965109181, 3028205659, 1332566561, 1443494453, 3972264077, 2796138953, 1811808701, 3995829011, 3264371483, 3284732653, 2252592907, 2944218709, 1195921303, 1240390289, 3338920877, 2016761807, 2529498863, 952181821, 4333778233, 4693567763, 4688538703, 4905109823, 4043754641, 2731077191, 3889851109, 2314277087, 4110023501, 2548368847, 3368441227, 1232754433, 4611307861, 1024710161, 3398593411, 3489896321};
\end{lstlisting}
\subsection{BITSET}
\begin{lstlisting}
bitset<size> variable_name;
b1; b1.set(3); // Set the 3rd bit to 1
b2(255); b2.reset(7); // Reset the 7th bit to 0
b3(5); b3.flip(2); // Flip the 2nd bit
b4(42); b4.count(); // Count the number of set bits
b5(5); b5.test(2); // Check if the 2nd bit is set
b6(0); b6.any(); // Check if any bit is set
b7(0); b7.none(); // Check if no bit is set
b8(255); b8.all(); // Check if all bits are set
b9(1); b9 <<= 2; // Left shift by 2
b10(4); b10 >>= 1; // Right shift by 1
&lt; and &gt; instead of < and >
\end{lstlisting}

\subsection{Ternary Search}
\begin{lstlisting}
// Ternary search to find maximum of a unimodal function f in [l,r]
double ternary_search(double l,double r){
    double eps=1e-9; // error tolerance
    while(r-l>eps){
        double m1=l+(r-l)/3;
        double m2=r-(r-l)/3;
        double f1=f(m1); // evaluate function at m1
        double f2=f(m2); // evaluate function at m2
        if(f1<f2) l=m1;
        else r=m2;
    }
    return f(l); // returns maximum of f(x) in [l,r]
}
\end{lstlisting}



\end{multicols}

\end{document}
